\textbf{Stein, Shakarchi 2.15}
Suppose $f$ is a non-vanishing continuous function on $\closeD$ which is holomorphic in $\mathbb{D}$. Prove
that if

$$
\left| f(z) \right| = 1 \quad \text{whenever} \; |z| = 1,
$$

then $f$ is constant.

\begin{solution}
    Consider the function

    $$
    \tilde{f}(z) = \begin{cases}
        f(z), &|z| \le 1, \\
        1 / \overline{f(\sfrac{1}{\overline{z}})}, &|z| > 1.
    \end{cases}
    $$

    Our aim is to show that $\tilde{f}$ is entire and bounded so that it (and therefore 
    $f = \tilde{f}\vert_{\closeD}$) is constant by Liouville's Theorem. We first show that $\tilde{f}$ is 
    continuous on $\mathbb{C}$. When $|z| > 1$ (i.e., $z, \bar{z} \in \mathbb{C} \setminus \overline{\mathbb{D}})$, we have

    $$
    \left| \frac{1}{\bar{z}} \right| = \frac{1}{|z|} < 1,
    $$

    so that $\overline{f(\sfrac{1}{\overline{z}})}$ is never zero (for if $\overline{f(\sfrac{1}{\overline{z_0}})} = 0$ 
    for some $|z_0| > 1$, we would have 
    $0 = \overline{0} = \overline{\overline{f(\sfrac{1}{\overline{z_0}})}} = f(\sfrac{1}{\overline{z_0}})$, which
    contradicts the fact that $f$ is non-vanishing on $\closeD$). We consider an arbitrary sequence $\left\{z_n\right\}$ 
    outside the disc $\closeD$ such that $|z_n| \to 1$ (and hence $\left|\overline{z_n}\right| \to 1$); the reciprocal 
    of this sequence corresponds to a sequence $\left\{ \sfrac{1}{z_n} \right\}$ entirely contained inside $\closeD$ 
    such that $\left| \sfrac{1}{z_n} \right| \to 1$ (and hence $\left| \sfrac{1}{\overline{z_n}} \right| \to 1$, where 
    $\sfrac{1}{\overline{z_n}} \in \closeD$). By continuity of $f$ on $\closeD$, we therefore have

    $$
    \lim_{\left| z_n \right| \to 1} \left| \frac{1}{\overline{f}(\sfrac{1}{\overline{z_n}})} \right|
    = \lim_{\left| z_n \right| \to 1} \left| \frac{1}{f(\sfrac{1}{\overline{z_n}})} \right|
    = \left| \frac{1}{1} \right|
    = 1.
    $$

    and hence $f(z)$ and $1 / \overline{f(\sfrac{1}{\overline{z}})}$ agree on the boundary of $\closeD$. Since 
    $\{z_n\}$ was an arbitrary sequence in $\mathbb{C} \setminus \closeD$, we conclude that $\tilde{f}$ is continuous
    on all of $\mathbb{C}$. 
    
    By definition, $\tilde{f}$ is holomorphic on the open disc $\mathbb{D}$. It remains to show that $\tilde{f}$ is 
    holomorphic on the boundary of the disc and on $\mathbb{C} \setminus \closeD$. We proceed with the latter. Since $f$ 
    is non-vanishing, we have that $\frac{1}{f(\sfrac{1}{z})}$ is holomorphic whenever $|z| > 1$ as a composition of 
    holomorphic functions . Let $z_0 \in \mathbb{C} \setminus \closeD$. Since $|z_0| > 1$, we observe that 
    $|\overline{z_0}| > 1$ and hence there exists a power series expansion 
    $\frac{1}{f(\sfrac{1}{z})} = \sum a_n (z - \overline{z_0})^n$ about $\overline{z_0}$ in some sufficiently small open 
    disc. We therefore find that for $z$ in that disc,

    $$
    \frac{1}{\overline{f(\sfrac{1}{\overline{z}})}}
    = \overline{\left(\frac{1}{f(\sfrac{1}{\overline{z}})} \right)} 
    = \overline{\sum a_n (\overline{z} - \overline{z}_0)^n}
    = \left( \sum \overline{a_n} \cdot \overline{(\overline{z} - \overline{z_0})^n} \right)
    = \sum \overline{a_n} (z - z_0)^n.
    $$

    so that a power series expansion evidently exists for $\frac{1}{\overline{f(\sfrac{1}{\overline{z}})}}$ about any 
    arbitrary $z_0 \in \mathbb{C} \setminus \closeD$ and so $\tilde{f}$ is holomorphic outside the unit disc.

    Lastly, to show that $\tilde{f}$ is entire, we consider a an arbitrary simple closed Jordan curve $\gamma$ which 
    intersects the circle $|z| = 1$ (any such curve which does not intersect this circle integrates to zero by 
    holomorphicity of $\tilde{f}$ away from this circle). By splitting the curve where it intersects the circle by some 
    arbitrary distance $\epsilon > 0$ (see Figure \ref{fig:problem_12}), we obtain two curves $\gamma_1$ and $\gamma_2$ 
    which are entirely contained in $\closeD$ and $\mathbb{C} \setminus \closeD$, respectively. Since $\tilde{f}$ is
    holomorphic in these regions, the integral of $\tilde{f}$ over each curve is zero.  Since this is true for every 
    $\epsilon > 0$, we let $\epsilon \to 0$ and conclude by continuity of $\tilde{f}$ over all $\mathbb{C}$ that the 
    integral of $\tilde{f}$ over $\gamma$ is also zero. Since $\gamma$ was an arbitrary simple closed Jordan curve, we 
    have by Morera's Theorem that $\tilde{f}$ is holomorphic on $|z| = 1$ and is therefore entire. Since $f$ is a 
    continuous function on the compact disc $\closeD$, it (and therefore its conjugate reciprocal) is bounded. By 
    Liouville's Theorem, we have that $\tilde{f}$ is constant as was desired.

    \vspace{1.0em}

    \begin{figure}[h]
        \centering
        \begin{subfigure}{0.45\textwidth}
            \includegraphics*[width=\textwidth]{problem_12_full_curve.png}
        \end{subfigure}
        \hspace{1.0em}
        \begin{subfigure}{0.45\textwidth}
            \includegraphics*[width=\textwidth]{problem_12_split_curve.png}
        \end{subfigure}
        \caption{Splitting a simple closed Jordan curve $\gamma$ which intersects $|z| = 1$}
        \label{fig:problem_12}
    \end{figure}

    \ \\
\end{solution}