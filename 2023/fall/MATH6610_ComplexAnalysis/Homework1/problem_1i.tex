Determine which of the following points are stationary points:

\begin{enumerate}
    \item $(0, 0, 2)^T$
    \item $(0, 0, 3)^T$
    \item $(1, 0, 1)^T$
\end{enumerate}

\begin{solution}
    We begin by computing the gradient of $f$:

    $$
    \nabla f = \begin{pmatrix*}
        2 x_1 + 2 x_1 x_3^2 + 2 x_2 \\
        2 x_1 + 4 x_2^3 + 8 \\
        2x_1^2 x_3
    \end{pmatrix*}.
    $$

    Our constraint matrix $A$ is given by $A = \begin{pmatrix*}2 &  5 & 1\end{pmatrix*}$ and hence the null space of $A$ 
    is given by all vectors which satisfy $2 x_1 + 5 x_2 + x_3 = 0$. From this, we obtain a null space matrix $Z$ 
    whose columns are the basis of the null space of $A$:

    $$
    Z = \begin{pmatrix*}[r]
        -2 & -1 \\
         1 & 0  \\
        -1 & 2
    \end{pmatrix*}.
    $$

    Stationary points must satisfy $Z^T \nabla(x_*) = 0$. We now check each of the given points. The point $(0, 0, 2)^T$
    is infeasible, though we still verify that it does not satisfy the stationary condition; the point $(0, 0, 3)^T$ is 
    feasible but not stationary, and the point $(1, 0, 1)$ is both feasible and stationary:\footnote{
        Matrix computations and gradient evalutations are performed in \texttt{problem\_1i.py}.
    }

    \begin{align*}
        Z^T \nabla f(0, 0, 2) &= \begin{pmatrix*}[r]
            -2 &  1 & -1 \\
            -1 &  0 &  2
        \end{pmatrix*} \begin{pmatrix*}[r]
            0 \\
            8 \\
            0
        \end{pmatrix*} = \begin{pmatrix*}[r]
            8 \\
            0
        \end{pmatrix*} > 0 \\
        Z^T \nabla f(0, 0, 3) &= \begin{pmatrix*}[r]
            -2 &  1 & -1 \\
            -1 &  0 &  2
        \end{pmatrix*} \begin{pmatrix*}[r]
            0 \\
            8 \\
            0
        \end{pmatrix*} = \begin{pmatrix*}[r]
            8 \\
            0
        \end{pmatrix*} > 0 \\
        Z^T \nabla f(1, 0, 1) &= \begin{pmatrix*}[r]
            -2 &  1 & -1 \\
            -1 &  0 &  2
        \end{pmatrix*} \begin{pmatrix*}
            4  \\
            10 \\
            2   
        \end{pmatrix*} = \begin{pmatrix*}[r]
            0 \\
            0
        \end{pmatrix*} = 0.
    \end{align*}
    \ \\
\end{solution}