$$
A = \begin{pmatrix*}[r]
  1 &  1 &  1 & 1 \\
  1 & -1 & -1 & 1 \\
  0 &  1 &  0 & 1
\end{pmatrix*},\; x_1 = \begin{pmatrix}
  1 \\
  3 \\
  1 \\
  2
\end{pmatrix},\; x_2 = \begin{pmatrix*}[r]
   0 \\
  -2 \\
  -3 \\
   4
\end{pmatrix*}
$$


\begin{solution}
  We observe first that the columns of $A^T$ are linearly independent, since the reduced-row echelon of $A^T$ takes the
  form

  $$
  \text{rref}(A^T) = \begin{pmatrix*}[r]
    I \\
    0
  \end{pmatrix*}.
  $$

  Hence $\text{rank}(A) = 3$, and so by the rank-nullity theorem, $\text{nullity}(A) = 1$. Moreover, the columns of 
  $A^T$ form a basis for the range space of $A^T$. The reduced-row echelon form of $A$ is


  $$
  \text{rref}(A^T) = \begin{pmatrix*}[r]
    1 & 0 & 0 & 1 \\
    0 & 1 & 0 & 1 \\
    0 & 0 & 1 & -1 \\
  \end{pmatrix*}.
  $$

  We let $\text{rref}(A)z = 0$ and observe that $z_4$ is the only free variable; hence a basis for the null space of
  $A$ is given by

  $$
  \beta_{N(A)} = \left\{ \beta_1 \right\} = \left\{\begin{pmatrix*}[r]
    -1 \\
    -1 \\
     1 \\
     1
  \end{pmatrix*}\right\}.
  $$

  To compute each point $x_i$ as the sum of vectors in the null and row spaces, we project each $x_i$ onto the 
  null space basis vector and subtract out the result; what remains is the component of $x_i$ which resides in the row 
  space. \footnote{
    This is a result of the fact that the row space and null space are orthogonal complements of each other and, 
    crucially, that the null space basis consists only of a single vector. If the null space consists of multiple 
    vectors, a Gram-Schmidt process would be required to compute the rejection of $x_i$ from the 
    null space.
  } \footnote{
    Computations are performed in \texttt{problem\_7i.py}.
  }

  For $x_1 = (1, 3, 1, 2)^T$, we have:

  $$
  p_1 = \text{proj}_{\beta_1} x_1 \cdot \beta_1 = -\frac{1}{4} \beta_1 = \begin{pmatrix*}[r]
    \frac{1}{4} \\[6pt]
    \frac{1}{4} \\[6pt]
   -\frac{1}{4} \\[6pt]
   -\frac{1}{4}
  \end{pmatrix*}
  $$

  and hence

  $$
  q_1 = x_1 - p_1 
      = \begin{pmatrix*}[r]
        1 \\[6pt]
        3 \\[6pt]
        1 \\[6pt]
        2
      \end{pmatrix*}
      - \begin{pmatrix*}[r]
        \frac{1}{4} \\[6pt]
        \frac{1}{4} \\[6pt]
       -\frac{1}{4} \\[6pt]
       -\frac{1}{4}
      \end{pmatrix*}
      = \begin{pmatrix*}[r]
        \frac{3}{4} \\[6pt]
        \frac{11}{4} \\[6pt]
        \frac{5}{4} \\[6pt]
        \frac{9}{4}
      \end{pmatrix*}.
  $$

  Similarly, for $x_2 = (0, -2, -3, 4)^T$, we have:

  $$
  p_2 = \text{proj}_{\beta_1} x_2 \cdot \beta_1 = \frac{3}{4} \beta_1 = \begin{pmatrix*}[r]
   -\frac{3}{4} \\[6pt]
   -\frac{3}{4} \\[6pt]
    \frac{3}{4} \\[6pt]
    \frac{3}{4}
  \end{pmatrix*}
  $$

  and so

  $$
  q_2 = x_2 - p_2 
      = \begin{pmatrix*}[r]
        0 \\[6pt]
       -2 \\[6pt]
       -3 \\[6pt]
        4
      \end{pmatrix*}
      - \begin{pmatrix*}[r]
       -\frac{3}{4} \\[6pt]
       -\frac{3}{4} \\[6pt]
        \frac{3}{4} \\[6pt]
        \frac{3}{4}
      \end{pmatrix*}
      = \begin{pmatrix*}[r]
        \frac{3}{4} \\[6pt]
       -\frac{5}{4} \\[6pt]
       -\frac{15}{4} \\[6pt]
        \frac{13}{4}
      \end{pmatrix*}.
  $$

  \ \\
\end{solution}