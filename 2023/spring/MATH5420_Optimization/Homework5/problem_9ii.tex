What would the new solution be if the right-hand side of the third constraint were increased by 5?

\begin{solution}
  \ \\
  Let $\Delta b = (0, 0, 5)^T$. We first verify that the new solution is feasible. As in (i), we observe

  $$
  B^{-1}\bar{b} = B^{-1} \left( b + \Delta b \right) = B^{-1}b + B^{-1}\Delta b = \begin{pmatrix*}[r]
    5 \\
    3 \\
    3 
  \end{pmatrix*} + \begin{pmatrix*}[r]
    5 \\
    0 \\
    0
  \end{pmatrix*} \ge \begin{pmatrix*}[r]
    0 \\
    0 \\
    0
  \end{pmatrix*}
  $$

  and hence the new solution is feasible. To determine the new solution, we recall from the results of Example 6.14 that

  $$
  y^T = c_B^T B^{-1} = \begin{pmatrix*}[r]
    0 & -1 & -2
  \end{pmatrix*}.
  $$

  Our new optimal value $\bar{z}$ is therefore given by: \footnote{
    The resulting values for $x_B$ are computed in \texttt{problem\_9ii.py} for the interested reader.
  }
  
  \begin{align*}
    \bar{z} &= c_B^T B^{-1} \bar{b} \\
            &= y^T \left(b + \Delta b \right) \\
            &= y^T b + y^T \Delta b \\
            &= (0, -1, -2)(2, 7,  3)^T + (0, -1, -2)(0, 0, 5)^T \\
            &= -13 - 10 \\
            &= -23.
  \end{align*}
  \ \\
\end{solution}
