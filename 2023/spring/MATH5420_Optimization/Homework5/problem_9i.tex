By how much can the right-hand side of the first constraint change before the current basis ceases to be optimal?

\begin{solution}
  From Example 6.14, we have:

  $$
  B = \begin{pmatrix*}[r]
    1 & -2 & 1 \\
    2 & -1 & 0 \\
    0 &  1 & 0
  \end{pmatrix*}, \qquad B^{-1} = \begin{pmatrix*}[r]
    1 &  \sfrac{1}{2} & \sfrac{1}{2} \\
    0 &             0 &            1 \\
    1 & -\sfrac{1}{2} & \sfrac{3}{2} \\
  \end{pmatrix*}, \qquad b = \begin{pmatrix*}[r]
    2 \\
    7 \\
    3
  \end{pmatrix*}, \qquad B^{-1} b = \begin{pmatrix*}[r]
    5 \\
    3 \\
    3
  \end{pmatrix*}
  $$

  with optimal basis $x_B = \left( x_2, x_1, x_3 \right)^T$. Since modifying the problem's constraints has no effect on
  the optimality condition, we need only consider the perturbative effects on the feasibility of the optimal basis. The
  current basis will remain optimal so long as $B^{-1}b \ge 0$; hence we define $\bar{b} = b + \Delta b$ where 
  $\Delta b = (\delta, 0, 0)^T$ and observe that
  
  $$
  B^{-1}\bar{b} = B^{-1} \left( b + \Delta b \right) = B^{-1}b + B^{-1}\Delta b = \begin{pmatrix*}[r]
    5 \\
    3 \\
    3 
  \end{pmatrix*} + \begin{pmatrix*}[r]
    0 \\
    0 \\
    \delta
  \end{pmatrix*}.
  $$

  Since $B^{-1}\bar{b} \ge 0$ so long as $\delta \ge -3$, the current basis remains feasible so long as $b_1 \ge -1$.
  \ \\
\end{solution}
