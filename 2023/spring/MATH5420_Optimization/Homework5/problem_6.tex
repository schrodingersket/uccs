\textbf{Griva, Nash, Sofer 6.2.3}

Prove that if the primal is unbounded, then the dual is infeasible, and vice versa.

\begin{solution}
  Suppose first that the primal problem (in standard form) is unbounded. For the sake of contradiction, suppose there
  exists a feasible solution $y$ to the dual problem. By weak duality, $b^T y \le c^T x$ for all feasible points $x$ in 
  the primal, and so $b^T y$ is a lower bound for the primal problem. This contradicts our assumption that the primal 
  (minimization) problem is unbounded, and so the dual is infeasible.

  A similar argument holds for the converse: suppose that the dual of the primal (again in standard form) is unbounded 
  and for the sake of contradiction, suppose there exists a feasible solution $x$ to the primal problem. By weak 
  duality, $c^T x \ge b^T y$ for all feasible solutions $y$ to the dual, and so $c^T x$ is an upper bound for the dual 
  problem.  This contradicts our assumption that the dual (maximization) problem is unbounded, and so the primal problem 
  is infeasible.
  \ \\
\end{solution}
