\textbf{Griva, Nash, Sofer 5.4.1}

Use the simplex method (via a phase-1 problem) to find a basic feasible solution to the following system of linear 
inequalities:

\begin{align*}
  2x_1 - 3x_2 + 2x_3  & \ge 3 \\
  -x_1 +  x_2 +  x_3  & \ge 5 \\
  x_i                 & \ge 0.
\end{align*}

\begin{solution}
  We begin by converting the problem to standard form by introducing excess variables $x_4$ and $x_5$:

  \begin{align*}
    2x_1 - 3x_2 + 2x_3 - x_4 & = 3 \\
    -x_1 +  x_2 +  x_3 - x_5 & = 5 \\
    x_i                      & \ge 0.
  \end{align*}

  Were we to allow $x_4$ and $x_5$ to be a part of our entry basis, these variables would necessarily take infeasible 
  values $x_4 = -3$ and $x_5 = -5$. To determine a basic feasible solution, we therefore introduce artificial variables 
  $a_1$ and $a_2$ along with an auxiliary objective function:

  \begin{mini*}
    {}{z' = a_1 + a_2}{}{}
    \addConstraint{2x_1 - 3x_2 + 2x_3 - x_4 + a_1}{= 3}
    \addConstraint{-x_1 + x_2 + x_3 - x_5 + a_2}{= 5}
    \addConstraint{x_i, a_i}{\geq 0}.
  \end{mini*}

  Two iterations of the simplex method for this program yields our desired initial basis $x_B = (x_3, x_2)^T$ which
  corresponds to the basic feasible solution $x = \left(0, \sfrac{7}{5}, \sfrac{18}{5}, 0, 0\right)^T$ and yields
  $z' = 0$, as desired.\footnote{
    Unless otherwise noted, all simplex method problems are solved by way of \texttt{simplex.py} which may be found at
    \url{https://github.com/schrodingersket/uccs-coursework/blob/main/2023/spring/MATH5420_Optimization/Homework5/simplex.py}
    along with all other code written for this assignment. For each problem, this function is called from a suitably 
    named script; for example, this problem is solved by \texttt{problem\_1.py}.
  }
  \ \\
\end{solution}
