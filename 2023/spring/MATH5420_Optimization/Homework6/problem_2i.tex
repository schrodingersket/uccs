Write the first-order necessary condition. When does a stationary point exist?

\begin{solution}
    The first-order necessary condition for a stationary point of $f$ to exist requires that $\nabla f(x) = 0$ for some
    $x$. For a quadratic function, the gradient is given by

    $$
    \nabla f(x) = \frac{1}{2}\left(Q + Q^T\right)x + c^T.
    $$

    A stationary point therefore exists whenever the system

    $$
    \frac{1}{2}\left(Q + Q^T\right)x = -c^T.
    $$

    has a solution.\footnote{
        In the case that we require a unique stationary point, we further require that $Q + Q^T$ be nonsingular 
        (and therefore full rank).
    }
    \ \\
\end{solution}
