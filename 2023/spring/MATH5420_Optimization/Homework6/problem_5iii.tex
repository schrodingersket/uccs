Prove that there exists a $K \in \mathbb{N}$ such that

$$
\left\lVert x_{k+1} - x_* \right\rVert \le 
    L \left\lVert \nabla^2 f(x_*)^{-1} \right\rVert \left\lVert x_k - x_* \right\rVert^2
$$

whenever $k > K$ and therefore that Newton's method converges quadratically.

\begin{solution}
    Since $\nabla^2 f(x_*)$ is positive definite, its inverse $\nabla^2 f(x_*)^{-1}$ exists and is also positive 
    definite (since its spectrum is strictly positive, zero is not an eigenvalue and the reciprocals of all eigenvalues
    are also strictly positive). By part (ii), we need only show
    $$
        \frac{1}{2} \left\lVert \nabla^2 f(x_k)^{-1} \right\rVert \le \left\lVert \nabla^2 f(x_*)^{-1} \right\rVert
    $$

    in order to prove our claim. By the Triangle Inequality, we observe:

    \begin{align*}
    \left\lVert \nabla^2 f(x_k)^{-1} \right\rVert &= \left\lVert \nabla^2 f(x_k)^{-1} - \nabla^2 f(x_*)^{-1} + \nabla^2 f(x_*)^{-1} \right\rVert \\
                                                  &\le \left\lVert \nabla^2 f(x_k)^{-1} - \nabla^2 f(x_*)^{-1} \right\rVert + \left\lVert \nabla^2 f(x_*)^{-1} \right\rVert 
    \end{align*}

    Moreover, since the operator norm of matrix is equal to its largest eigenvalue, we have 
    $\left\lVert \nabla^2 f(x_*)^{-1} \right\rVert > 0$. Since $\nabla^2 f:\mathbb{R}^n \to \mathbb{R}^n$ is continuous 
    and invertible near $x_*$, $\nabla^2 f^{-1}$ is also continuous near $x_*$ (by invariance of domain) and so we we 
    may choose $K \in \mathbb{N}$ sufficiently large so that 
    $\left\lVert \nabla^2 f(x_k)^{-1} - \nabla^2 f(x_*)^{-1} \right\rVert < \left\lVert \nabla^2 f(x_*)^{-1} \right\rVert$
    whenever $k > K$. We therefore see that 

    \begin{align*}
        \frac{1}{2} \left\lVert \nabla^2 f(x_k)^{-1} \right\rVert &\le \frac{1}{2} \left\lVert \nabla^2 f(x_*)^{-1} \right\rVert + \frac{1}{2} \left\lVert \nabla^2 f(x_*)^{-1} \right\rVert \\
                                                                  &\le \left\lVert \nabla^2 f(x_*)^{-1} \right\rVert
    \end{align*}

    and hence

    \begin{align*}
        \left\lVert x_{k+1} - x_* \right\rVert &\le \frac{L}{2} \left\lVert \nabla^2 f(x_k)^{-1} \right\rVert \left\lVert x_k - x_* \right\rVert^2 \\
                                               &\le L \left\lVert \nabla^2 f(x_*)^{-1} \right\rVert \left\lVert x_k - x_* \right\rVert^2.
    \end{align*}

    \ \\
\end{solution}
