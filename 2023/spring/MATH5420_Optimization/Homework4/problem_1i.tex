\begin{mini*}
  {}{z = -5x_1 - 7x_2 - 12x_3 + x_4}{}{}
  \addConstraint{2x_1 + 3x_2 + 2x_3 + x_4}{\le 38}
  \addConstraint{3x_1 + 2x_2 + 4x_3 - x_4}{\le 55}
  \addConstraint{x_1, x_2, x_3, x_4}{\ge 0}.
\end{mini*}

\begin{solution}
  All computations for this problem are performed in \texttt{problem\_1i.py}; hence we only record the results of each
  iteration step of the simplex algorithm. We begin by representing this system in standard form as $Ax = b$, where

  $$
  A = \begin{pmatrix*}[r]
    2 & 3 & 2 &  1 & 1 & 0\\
    3 & 2 & 4 & -1 & 0 & 1 \\
  \end{pmatrix*}, \quad b = \begin{pmatrix*}[r]
    38 \\
    55
  \end{pmatrix*}, \quad x = \begin{pmatrix*}[r]
    x_1 \\
    x_2 \\
    x_3 \\
    x_4 \\
    x_5 \\
    x_6
  \end{pmatrix*}, \quad c = \begin{pmatrix*}[r]
    -5  \\
    -7  \\
    -12 \\
     1  \\
     0  \\ 
     0
  \end{pmatrix*}.
  $$

  \subsubsection*{Iteration 1}
  We begin by using the slack variables as our initial basis $x_B = \{x_5, x_6\}$ so that 
  $x_N = \{x_1, x_2, x_3, x_4\}$ and

  $$
    B = \begin{pmatrix*}[r]
      1 & 0 \\
      0 & 1
    \end{pmatrix*}, \quad N = \begin{pmatrix*}[r]
      2 & 3 & 2 &  1 \\
      3 & 2 & 4 & -1
    \end{pmatrix*}, \quad c_B = \begin{pmatrix*}[r]
      0 \\
      0
    \end{pmatrix*}, \quad c_N = \begin{pmatrix*}[r]
     -5  \\
     -7  \\
     -12 \\
      1
    \end{pmatrix*}.
  $$

  Hence our reduced cost vector is given by 

  $$
  \hat{c}_N^T = c_N^T - c_B^T B^{-1} N = \begin{pmatrix*}[r]
    -5 & -7 & -12 & 1
  \end{pmatrix*}
  $$

  and so the third nonbasic variable $x_3$ enters the basis. Moreover, we have
  
  $$
  \hat{A}_3 = B^{-1} A_3 = \begin{pmatrix*}[r]
    2  \\
    4
  \end{pmatrix*}, \quad \hat{b} = B^{-1} b = \begin{pmatrix*}[r]
    38  \\
    55
  \end{pmatrix*}.
  $$

  and since $\frac{\hat{b}_2}{\hat{a}_{2,3}} = \frac{55}{4}$ is the minimum positive ratio, the second basic variable 
  $x_6$ leaves the basis.

  \subsubsection*{Iteration 2}
  At this iteration, $x_B = \{x_5, x_3\}$ and thus $x_N = \{x_1, x_2, x_4, x_6\}$.\footnote{
    We order variables according to the output of the \texttt{problem1\_i.py} script.
  }
  Our matrices for this iteration are

  $$
    B = \begin{pmatrix*}[r]
      1 & 2 \\
      0 & 4
    \end{pmatrix*}, \quad N = \begin{pmatrix*}[r]
      2 & 3 &  1 & 0 \\
      3 & 2 & -1 & 1
    \end{pmatrix*}, \quad c_B = \begin{pmatrix*}[r]
      0 \\
     -12
    \end{pmatrix*}, \quad c_N = \begin{pmatrix*}[r]
     -5  \\
     -7  \\
      1  \\
      0
    \end{pmatrix*}.
  $$

  Our reduced cost vector is therefore given by 

  $$
  \hat{c}_N^T = c_N^T - c_B^T B^{-1} N = \begin{pmatrix*}[r]
     4 & -1 & -2 & 3
  \end{pmatrix*}
  $$

  and so the third nonbasic variable $x_4$ enters the basis. Moreover, we have
  
  $$
  \hat{A}_4 = B^{-1} A_4 = \begin{pmatrix*}[r]
    \sfrac{3}{2}  \\
    \sfrac{-1}{4}
  \end{pmatrix*}, \quad \hat{b} = B^{-1} b = \begin{pmatrix*}[r]
    \sfrac{21}{2}  \\
    \sfrac{55}{4}
  \end{pmatrix*}
  $$

  and since $\frac{\hat{b}_1}{\hat{a}_{1,4}} = 7$ is the minimum positive ratio, the first basic variable $x_5$ leaves 
  the basis.

  \subsubsection*{Iteration 3}
  We now have $x_B = \{x_4, x_3\}$ and thus $x_N = \{x_1, x_2, x_5, x_6\}$. Our matrices for this iteration are

  $$
    B = \begin{pmatrix*}[r]
      1 & 2 \\
     -1 & 4
    \end{pmatrix*}, \quad N = \begin{pmatrix*}[r]
      2 & 3 &  1 & 0 \\
      3 & 2 &  0 & 1
    \end{pmatrix*}, \quad c_B = \begin{pmatrix*}[r]
      1 \\
     -12
    \end{pmatrix*}, \quad c_N = \begin{pmatrix*}[r]
     -5  \\
     -7  \\
      0  \\
      0
    \end{pmatrix*}.
  $$

  Our reduced cost vector is therefore given by 

  $$
  \hat{c}_N^T = c_N^T - c_B^T B^{-1} N = \begin{pmatrix*}[r]
     \sfrac{14}{3} & \sfrac{5}{3} & \sfrac{4}{3} & \sfrac{7}{3}
  \end{pmatrix*}
  $$

  Since every element of $\hat{c}_N^T$ is positive, this basis is optimal. Our optimal basic feasible solution is thus
  $\bar{x} = \begin{pmatrix*}[r]
    0 & 0 & \sfrac{31}{2} & 7 & 0 & 0
  \end{pmatrix*}$ which yields an optimal value of $z = -179$.
  \ \\
\end{solution}