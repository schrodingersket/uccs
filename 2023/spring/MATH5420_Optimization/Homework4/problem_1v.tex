\begin{maxi*}
  {}{z = 7x_1 + 8x_2}{}{}
  \addConstraint{4x_1 + x_2}{\le 100}
  \addConstraint{x_1 + x_2}{\le 80}
  \addConstraint{x_1}{\le 40}
  \addConstraint{x_1, x_2}{\ge 0}.
\end{maxi*}

\begin{solution}
  All computations for this problem are performed in \texttt{problem\_1v.py}; hence we only record the results of each
  iteration step of the simplex algorithm. We begin by representing this system in standard form as $Ax = b$, wherein we
  wish to minimize $\hat{z} = -z$ (so that the sign of $c$ changes):

  $$
  A = \begin{pmatrix*}[r]
    4 & 1 & 1 & 0 & 0 \\
    1 & 1 & 0 & 1 & 0 \\
    1 & 0 & 0 & 0 & 1 \\
  \end{pmatrix*}, \quad b = \begin{pmatrix*}[r]
    100 \\
     80 \\
     40
  \end{pmatrix*}, \quad x = \begin{pmatrix*}[r]
    x_1 \\
    x_2 \\
    x_3 \\
    x_4 \\
    x_5
  \end{pmatrix*}, \quad c = \begin{pmatrix*}[r]
    -7  \\
    -8  \\
     0  \\
     0  \\
     0
  \end{pmatrix*}.
  $$

  \subsubsection*{Iteration 1}
  We begin by using the slack variables as our initial basis $x_B = \{x_3, x_4, x_5\}$ so that 
  $x_N = \{x_1, x_2\}$ and

  $$
    B = \begin{pmatrix*}[r]
      1 & 0 & 0 \\
      0 & 1 & 0 \\
      0 & 0 & 1
    \end{pmatrix*}, \quad N = \begin{pmatrix*}[r]
      4 & 1 \\
      1 & 1 \\
      1 & 0
    \end{pmatrix*}, \quad c_B = \begin{pmatrix*}[r]
      0 \\
      0 \\
      0
    \end{pmatrix*}, \quad c_N = \begin{pmatrix*}[r]
     -7 \\
     -8
    \end{pmatrix*}.
  $$

  For the two-dimensional problem, this basis corresponds to the point $(x_1, x_2) = (0, 0)$, plotted in 
  Figure \ref{fig:problem_1v_iteration_1}. Our reduced cost vector is given by 

  \begin{figure}[h]
    \centering
    \includegraphics*[width=0.85\textwidth]{problem_1v_1.png}
    \caption{Feasible Region at $x_B = \{x_3, x_4, x_5 \}$}
    \label{fig:problem_1v_iteration_1}
  \end{figure}

  $$
  \hat{c}_N^T = c_N^T - c_B^T B^{-1} N = \begin{pmatrix*}[r]
    -7 & -8
  \end{pmatrix*}
  $$

  and so the second nonbasic variable $x_2$ enters the basis. Moreover, we have
  
  $$
  \hat{A}_2 = B^{-1} A_2 = \begin{pmatrix*}[r]
    1  \\
    1  \\
    0
  \end{pmatrix*}, \quad \hat{b} = B^{-1} b = \begin{pmatrix*}[r]
    100 \\
     80 \\
     40
  \end{pmatrix*}.
  $$

  and since $\frac{\hat{b}_2}{\hat{a}_{2,2}} = 80$ is the minimum positive ratio, the second basic variable 
  $x_4$ leaves the basis.

  \pagebreak
  \subsubsection*{Iteration 2}
  At this iteration, $x_B = \{x_3, x_2, x_5\}$ and thus $x_N = \{x_1, x_4\}$. Our matrices for this iteration are

  $$
    B = \begin{pmatrix*}[r]
      1 & 1 & 0 \\
      0 & 1 & 0 \\
      0 & 0 & 1
    \end{pmatrix*}, \quad N = \begin{pmatrix*}[r]
      4 & 0 \\
      1 & 1 \\
      1 & 0
    \end{pmatrix*}, \quad c_B = \begin{pmatrix*}[r]
      0 \\
     -8 \\
      0
    \end{pmatrix*}, \quad c_N = \begin{pmatrix*}[r]
     -7  \\
      0
    \end{pmatrix*}.
  $$

  In the two-dimensional case, this basis corresponds to the point $(x_1, x_2) = (0, 80)$ (plotted in 
  \ref{fig:problem_1v_iteration_2}). Our reduced cost vector is given by 

  $$
  \hat{c}_N^T = c_N^T - c_B^T B^{-1} N = \begin{pmatrix*}[r]
     1 & 8
  \end{pmatrix*}.
  $$

  \begin{figure}[h]
    \centering
    \includegraphics*[width=0.85\textwidth]{problem_1v_2.png}
    \caption{Feasible Region at $x_B = \{x_3, x_2, x_5 \}$}
    \label{fig:problem_1v_iteration_2}
  \end{figure}
 
  \pagebreak
  Since every element of $\hat{c}_N^T$ is positive, this basis is optimal. Our optimal basic feasible solution is thus
  $\bar{x} = \begin{pmatrix*}[r]
    0 & 80 & 20 & 0 & 40
  \end{pmatrix*}$ which yields an optimal value of $z = -\hat{z} = 640$.
  \ \\
\end{solution}