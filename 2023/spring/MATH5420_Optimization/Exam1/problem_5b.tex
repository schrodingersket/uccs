Use the phase-I method to obtain a basic feasible solution.

\begin{solution}

    We begin by converting our system from (a) into standard form by introducing the slack variable $x_3'$:

    \begin{mini*}
        {}{\tilde{z} = x_1' - x_2'}{}{}
        \addConstraint{x_1' + x_2' + x_3'}{= 2}
        \addConstraint{x_1', x_2', x_3'}{\ge 0}.
    \end{mini*}

    To use the phase-I method to obtain a basic feasible solution, we add a new artificial variable $a_4 \ge 0$ along 
    with a new objective function to obtain the auxiliary system:\footnote{
        Since we already introduced a single slack variable to our single-constraint system, we could have simply chosen
        $x_B = \{ x_3' \}$ as our initial basis and called it a day. Such a choice does, however, obviate the need for
        the two-phase method and is simply not as much fun.
    }

    \begin{mini*}
        {}{\tilde{z}' = a_4}{}{}
        \addConstraint{x_1' + x_2' + x_3' + a_4}{= 2}
        \addConstraint{x_1', x_2', x_3', a_4}{\ge 0}.
    \end{mini*}

    We solve this system via the simplex method in \texttt{problem\_5b.py} with initial basis $x_B = \{ a_4 \}$. At the
    first iteration, our simplex quantities of interest are given by

    \begin{align*}
      B &= \begin{pmatrix}
        1  \\
      \end{pmatrix}, \quad y = \begin{pmatrix}
        1 
      \end{pmatrix}^T, \quad c_B = \begin{pmatrix}
        1
      \end{pmatrix}^T \\
      \hat{b} &= \begin{pmatrix}
        2
      \end{pmatrix}^T, \quad \hat{c}_N = \begin{pmatrix}
        -1 & -1 & -1
      \end{pmatrix}^T, \quad \tilde{z}' = 2.
    \end{align*}

    Since all our reduced cost variables are negative and identical, we arbitrarily choose the basic variable $x_1$ as
    our new entering variable. Since our constraint ratio vector is simply the scalar 2, we choose $a_4$ as the leaving 
    variable so that our next simplex iteration yields:

    \begin{align*}
      B &= \begin{pmatrix}
        1  \\
      \end{pmatrix}, \quad y = \begin{pmatrix}
        0 
      \end{pmatrix}^T, \quad c_B = \begin{pmatrix}
        0
      \end{pmatrix}^T \\
      \hat{b} &= \begin{pmatrix}
        2
      \end{pmatrix}^T, \quad \hat{c}_N = \begin{pmatrix}
        0 & 0 & 1
      \end{pmatrix}^T, \quad \tilde{z}' = 0.
    \end{align*}

    Our initial feasible basis for the primal is therefore $x_B = \{ x_1 \}$.

\end{solution}