Find the optimal basic feasible solution using a tableau with the origin as the initial guess.

\begin{solution}
  With $x_B = \{ x_4, x_5 \}$ as our initial basic feasible solution (corresponding to slack variables), our tableau
  takes the form:

  \[\arraycolsep=8pt\def\arraystretch{2.2}
  \begin{array}{r|rrrrr|r|r}
     \text{Basic} &  x_1  &  \bm{x_2}  &  x_3  &  x_4  & x_5 &  \text{RHS} & \text{(ratio)}  \\ \hline
    -z            &  -6   &  -14       &  -13  &   0   &  0  &   0         & 0               \\ \hline
     \bm{x_4}     &   1   &   4        &   2   &   1   &  0  &  48         & 12              \\
     x_5          &   1   &   2        &   4   &   0   &  1  &  60         & 30
  \end{array}
  \]

  Since $c_N^T$ contains negative entries, this feasible solution is not optimal, and we choose $x_2$ (the most negative
  entry of $c_N^T$) as our entering variable. We compute the ratios of the RHS column to the column of $A$ corresponding
  to $x_2$, select the basic variable with the smallest positive ratio as our exiting variable ($x_4$ in this 
  iteration), and denote these entering and exiting variables in bold above in the "Basic" row and column, 
  respectively. We pivot on the $x_2$ column by applying the following sequential row operations:

  $$
  -\frac{1}{2} R_{x_4} + R_{x_5}  \longrightarrow \frac{7}{2} R_{x_4} + R_{-z} \longrightarrow \frac{1}{4} R_{x_4}
  $$

  which yields the tableau
  
  \[\arraycolsep=8pt\def\arraystretch{2.2}
  \begin{array}{r|rrrrr|r|r}
     \text{Basic} &  x_1           &  x_2 &  \bm{x_3}      &   x_4           &   x_5 & \text{RHS}  & \text{(ratio)}  \\ \hline
     -z           &  \sfrac{-5}{2} &  0   &  -6            &   \sfrac{7}{2}  &   0   &  168        &                 \\ \hline
      x_2         &  \sfrac{1}{4}  &  1   &   \sfrac{1}{2} &   \sfrac{1}{4}  &   0   &  12         & 24              \\
     \bm{x_5}     &  \sfrac{1}{2}  &  0   &   3            &   \sfrac{-1}{2} &   1   &  36         & 12              
  \end{array}
  \]

  \ \\
  We select $x_3$ as our new entering variable, $x_5$ as our exiting variable, and pivot about the $x_3$ column via 
  row operations

  $$
  -\frac{1}{6} R_{x_5} + R_{x_2}  \longrightarrow 2 R_{x_5} + R_{-z} \longrightarrow \frac{1}{3} R_{x_5}
  $$

  to obtain our next tableau:

  \[\arraycolsep=8pt\def\arraystretch{2.2}
  \begin{array}{r|rrrrr|r|r}
     \text{Basic} &  \bm{x_1}      &  x_2 &  x_3  &  x_4           &  x_5           &  \text{RHS} &  \text{(ratio)} \\ \hline
    -z            &  \sfrac{-3}{2} &  10  &  0    &  \sfrac{5}{2}  &  2             &  240        &                 \\ \hline
     \bm{x_2}     &  \sfrac{1}{6}  &  1   &  0    &  \sfrac{1}{3}  &  \sfrac{-1}{6} &  6          &  36             \\
     x_3          &  \sfrac{1}{6}  &  0   &  1    &  \sfrac{-1}{6} &  \sfrac{1}{3}  &  12         &  72                           
  \end{array}
  \]

  We choose $x_1$ to enter the basis, $x_2$ to exit the basis, and pivot on the $x_1$ column via the following row 
  operations:

  $$
  -R_{x_2} + R_{x_3}  \longrightarrow 9 R_{x_2} + R_{-z} \longrightarrow 6 R_{x_2}
  $$


  which yields our final tableau:

  \[\arraycolsep=8pt\def\arraystretch{2.2}
  \begin{array}{r|rrrrr|r|r}
    \text{Basic} &  x_1  &  x_2 &  x_3  &  x_4           &  x_5           &  \text{RHS} \\ \hline
    -z           &  0    &  19  &  0    &  \sfrac{11}{2} &  \sfrac{1}{2}  &  294        \\ \hline
    x_1          &  1    &  6   &  0    &  2             & -1             &  36         \\
    x_3          &  0    & -1   &  1    &  \sfrac{-1}{2} &  \sfrac{1}{2}  &  6                            
  \end{array}
  \]

  Since all elements of $c_N$ are positive, we have arrived at our optimal basic feasible solution, which corresponds to
  the basis $x_B = \{x_1, x_3\}$. This point lies at \linebreak
  $\bar{x} = \begin{pmatrix} 36 &  0  &  6  &  0  &  0 \end{pmatrix}^T$ and yields an optimal value of $z = -294$ which
  agrees with the results of part (b).
  \ \\
\end{solution}