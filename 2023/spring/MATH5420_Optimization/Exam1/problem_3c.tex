Would adding an additional hour of available labor increase profit? If so, what is the new maximal profit? Justify these
conclusions with a sensitivity analysis.

\begin{solution}
  We consider the effect of perturbing the right-hand side of our constraint equations by $\Delta b = (1, 0)$ so that
  our new constraints are 

  $$
  \bar{b} = b + \Delta b = \begin{pmatrix} 12 \\ 500 \end{pmatrix}
  $$

  Since modifying the problem's constraints has no effect on whether a particular feasible basis stays optimal, we need
  only consider the effect of perturbation on the feasibility of the basis. The optimal basis determined in part (b) 
  will stay optimal for some new $\bar{b}$ so long as $B^{-1} \bar{b} \ge 0$. This quantity is computed in 
  \texttt{problem\_3.py} and is given by:

  $$
  B^{-1} \bar{b} = \begin{pmatrix} 
    2  & 3    \\
    50 & 150
  \end{pmatrix}^{-1} \begin{pmatrix} 
    12  \\
    500 
  \end{pmatrix} = \begin{pmatrix} 
    2  \\
    \frac{8}{3}
  \end{pmatrix} \ge 0
  $$

  and hence $x_B = \{ x_1, x_2 \}$ remains the optimal basis when we add an additional hour of available assembly labor.
  To determine the new revenue amount, we first compute

  $$
  y^T = c_B^T B^{-1} = \begin{pmatrix*}[r]
    -100 & -200
  \end{pmatrix*} \begin{pmatrix}
    2  & 3    \\
    50 & 150
  \end{pmatrix}^{-1} = \begin{pmatrix}
    \sfrac{-100}{3} & \sfrac{-2}{3}
  \end{pmatrix}
  $$

  so that our new optimal value $\bar{z}$ becomes: 
  
  \begin{align*}
    \bar{z'} &= c_B^T B^{-1} \bar{b} \\
             &= y^T \left(b + \Delta b \right) \\
             &=\begin{pmatrix}
                 \sfrac{-100}{3} & \sfrac{-2}{3}
               \end{pmatrix} \begin{pmatrix}
                 12  \\
                 500
               \end{pmatrix} \\
             &\approx -733.33.
  \end{align*}

  Assuming the bike shop is willing to partake in partial assembly, the new maximal revenue is 
  $\bar{z} = -\bar{z}' = \$733.33$. Adding an extra hour of available labor therefore increases the total revenue by 
  \$33.33. Assuming we may split parts costs in the same way as bike assembly, the new profit is 
  $\bar{z} - 2(\$50) - 2.67(\$150) = \$733.33 - \$500 = \$233.33$.\footnote{
    We may instead choose to optimize for profit directly with the additional constraints $x_1, x_2 \ge 1$ (since at 
    least one of each bike must be made) and objective function $p = (100 - 50)x_1 + (200 - 150)x_2$. The simplex 
    algorithm for this system yields the optimal basis $x_B = \{ x_1, x_4 \}$; in this case, four basic bikes and one 
    deluxe bike should be assembled for a maximum profit of \$250.00. The simplex method for this problem is implemented
    in \texttt{problem\_3\_aux.py}.
  }

\end{solution}