Consider the coefficients

$$
Q = \begin{pmatrix}
    1 &  2 &  2 \\
    2 &  4 & -1 \\
    2 & -1 & 4
\end{pmatrix}, \quad c = \begin{pmatrix}
    1 \\
    1 \\
    1
\end{pmatrix}
$$

Is $Q$ positive definite? Apply one exact iteration of Newton's method using initial guess 
$x = \begin{pmatrix} 0 & 0 & 0 \end{pmatrix}^T$. Is this point the minimizer of $f(x)$? Discuss this result in relation 
to part (a).

\begin{solution}
    The spectrum of $Q$ is given by 

    $$
    \lambda(Q) = \{ -1, 5, 5 \}
    $$

    and hence $Q$ is not positive definite since not all eigenvalues are strictly positive.\footnote{
        See \texttt{problem\_4.py} for all matrix computations performed in this exercise.
    } We apply a single Newton iteration (noting that $\nabla^2 f = Q$ is nonsingular since 0 is not an eigenvalue of 
    $Q$) to obtain:

    $$
    x_1 = Q^{-1} c = \begin{pmatrix}
        -\sfrac{3}{5} &  \sfrac{2}{5} &  \sfrac{2}{5} \\
         \sfrac{2}{5} &  0            & -\sfrac{1}{5} \\
         \sfrac{2}{5} & -\sfrac{1}{5} &  0            \\
    \end{pmatrix} \begin{pmatrix}
        1 \\
        1 \\
        1
    \end{pmatrix} = \begin{pmatrix}
        \sfrac{1}{5} \\
        \sfrac{1}{5} \\
        \sfrac{1}{5}
    \end{pmatrix}.
    $$

    Substituting this into our expression for $\nabla f$ yields:

    $$
    \nabla f(x_1) = \begin{pmatrix}
        1 &  2 &  2 \\
        2 &  4 & -1 \\
        2 & -1 & 4
    \end{pmatrix} \begin{pmatrix}
        \sfrac{1}{5} \\
        \sfrac{1}{5} \\
        \sfrac{1}{5}
    \end{pmatrix} = \begin{pmatrix}
        0 \\
        0 \\
        0
    \end{pmatrix}
    $$

    which indicates that $x_1$ is a stationary point. Since the Hessian has a mix of positive and negative eigenvalues
    at every point $x$, it is indefinite and so $x_1$ represents a saddle point of $f$ which evaluates to 
    $f(x_1) = -0.3$. This point is not a minimizer for $f$, since evaluating $f$ at the nearby point 
    $x^* = (1.2, -0.8, -0.8 )^T$ yields 
    
    $$
    f(x^*) = -0.8 < -0.3 = f(x_1).
    $$

    In part (a), we showed that if $Q$ is invertible, Newton's method finds a stationary point in a single iteration. By
    dropping the assumption that $Q$ be positive definite, we lose the ability to guarantee that the first Newton
    iterate minimizes $f$ (which is consistent with the results of this exercise, since $x_1$ is a saddle point) but 
    retain the result that a stationary point is found in a single iteration.
    \ \\
\end{solution}