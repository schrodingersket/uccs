Find and classify all minimizers, maximizers, and saddle points.

\begin{solution}
    For convenience, we enumerate our constraints as follows:

    $$
    g_1(x) = -x_1^2 - x_2^2 + 1 \ge 0, \quad g_2(x) = x_1 + x_2 \ge 0.
    $$

    so that the gradients of our constraints are

    $$
    \nabla g_1 = \begin{pmatrix*}
        -2 x_1 \\
        -2 x_2
    \end{pmatrix*}, \quad \nabla g_2 = \begin{pmatrix*}
        1 \\
        1
    \end{pmatrix*}
    $$

    Since we have two constraints, we have two Lagrange multipliers and our Lagrangian is therefore

    $$
    \mathcal{L}(x, \lambda) = f(x) - \lambda_1 g_1 - \lambda_2 g_2 
                            = x_1^2 - x_2^2 - \lambda_1 \left[ -x_1^2 - x_2^2 + 1 \right] - \lambda_2 \left[ x_1 + x_2 \right]
    $$

    where the gradient of the Lagrangian with respect to $x$ is 

    $$
    \nabla_x \mathcal{L} = \nabla f(x) - \lambda_1 \nabla g_1 - \lambda_2 \nabla g_2 = \begin{pmatrix*}[r]
        2 x_1 \\
       -2 x_2
    \end{pmatrix*} - \lambda_1 \begin{pmatrix*}
       -2 x_1 \\
       -2 x_2
    \end{pmatrix*} - \lambda_2 \begin{pmatrix*}
        1 \\
        1
    \end{pmatrix*}
    $$

    with corresponding Hessian

    $$
    \nabla^2_{xx} \mathcal{L} = 2 \begin{pmatrix*}
        1 + \lambda_1 & 0 \\
                      0 & -1 + \lambda_1
    \end{pmatrix*}
    $$
    
    so that feasible stationary points are minimizers whenever $\lambda_1 \in (1, \infty)$, maximizers whenever 
    $\lambda_1 \in (-\infty, -1)$, saddle points when $\lambda \in (-1, 1)$, and degenerate when $\lambda = \pm 1$. We 
    now proceed to compute stationary points by enforcing our first-order necessary condition that 
    $\nabla_x \mathcal{L} = 0$ along with strict complementarity by cases.

    \paragraph{Case I:} $g_1 = g_2 = 0.$ \ \\
    When both constraints are active, we obtain the system of equations

    $$
    \begin{pmatrix*}[r]
        x_1 + x_2 \\
        -x_1^2 - x_2^2 + 1
    \end{pmatrix*} = \begin{pmatrix*}
        0 \\
        0
    \end{pmatrix*}
    $$

    From the first equation, we find $x_1 = -x_2$; substituting this into the second equation yields

    $$
    -x_1^2 - (-x_1)^2 + 1 = -2 x_1^2 + 1 = 0
    $$

    from which we obtain $x_1 = \pm \frac{1}{\sqrt{2}}$. For 
    $x_* = \left( \frac{1}{\sqrt{2}}, -\frac{1}{\sqrt{2}} \right)$, we compute $\lambda_1$ and $\lambda_2$ from the 
    system

    $$
    \begin{pmatrix*}
        \sqrt{2} + \sqrt{2} \lambda_1 - \lambda_2 \\
        \sqrt{2} - \sqrt{2} \lambda_1 - \lambda_2
    \end{pmatrix*} = \begin{pmatrix*}
        0 \\
        0
    \end{pmatrix*}
    $$

    which yields $\lambda_1 = 0$ and $\lambda_2 = \sqrt{2}$. At $x_*$, the Jacobian of the active constraint matrix is

    $$
    \nabla g(x_*) = \begin{pmatrix*}
        \vline        & \vline \\
        \nabla g_1(x) & \nabla g_2(x) \\
        \vline        & \vline
    \end{pmatrix*}^T \Bigg|_{x = x_*} = \begin{pmatrix*}
        -\sqrt{2} & \sqrt{2} \\
        1         & 1
    \end{pmatrix*}
    $$
   
    which has full rank and therefore a trivial null space. The point 
    $\left( \frac{1}{\sqrt{2}}, -\frac{1}{\sqrt{2}} \right)$ is therefore a local minimizer with objective value 
    $f(x) = 0$. Conversely, when $x_* = \left( -\frac{1}{\sqrt{2}}, \frac{1}{\sqrt{2}} \right)$, our first-order 
    necessary condition yields

    $$
    \begin{pmatrix*}
        -\sqrt{2} - \sqrt{2} \lambda_1 - \lambda_2 \\
        -\sqrt{2} + \sqrt{2} \lambda_1 - \lambda_2
    \end{pmatrix*} = \begin{pmatrix*}
        0 \\
        0
    \end{pmatrix*}
    $$

    from which we obtain $\lambda_1 = 0$ and $\lambda_2 = -\sqrt{2}$. The Jacobian of the active constraints $g_1$ 
    and $g_2$ at this point is 

    $$
    \nabla g(x_*) = \begin{pmatrix*}
        -\sqrt{2} & -\sqrt{2} \\
        1         & 1
    \end{pmatrix*}
    $$

    which is also full rank; since $\lambda_2 < 0$, the point $\left( -\frac{1}{\sqrt{2}}, \frac{1}{\sqrt{2}} \right)$
    is a local maximizer with objective value $f(x) = 0$.
    
    \paragraph{Case II:} $\lambda_1 = g_2 = 0.$ \ \\

    When $g_2$ is active, we have $x_1 = -x_2$; since $\lambda_1 = 0$, our first-order necessary condition becomes

    $$
    \nabla_x \mathcal{L} = \nabla f(x) - \lambda_2 \nabla g_2 = \begin{pmatrix*}[r]
        2 x_1 \\
       -2 x_2
    \end{pmatrix*} - \lambda_2 \begin{pmatrix*}
        1 \\
        1
    \end{pmatrix*} = \begin{pmatrix*}
        2 x_1 \\
        2 x_1
    \end{pmatrix*} - \lambda_2 \begin{pmatrix*}
        1 \\
        1
    \end{pmatrix*} = \begin{pmatrix*}
        0 \\
        0
    \end{pmatrix*}
    $$

    so that $\lambda_2 = 2 x_1 = -2 x_2$. We choose $x_1$ to be the free variable; upon substituting $x_2 = -x_1$ into 
    our inactive constraint $g_1 \ge 0$, we find that $-2 x_1^2 + 1 \ge 0$ so that the first-order necessary condition 
    is satisfied along the line $x_2 = -x_1$ for all $x_1 \in \left[ -\frac{1}{\sqrt{2}}, \frac{1}{\sqrt{2}} \right]$.

    The Jacobian of the active constraint $g_2$ is

    $$
    \nabla g_2 = \begin{pmatrix*}
        1 \\
        1
    \end{pmatrix*}
    $$

    which admits a null space matrix

    $$
    Z = \begin{pmatrix*}
         1 \\
        -1
    \end{pmatrix*}
    $$

    so that 

    $$
    Z^T \nabla_{xx}^2 \mathcal{L} Z = \begin{pmatrix*}
        1 & -1
    \end{pmatrix*} \begin{pmatrix*}[r]
        2 & 0 \\
        0 & -2
    \end{pmatrix*} \begin{pmatrix*}[r]
        1 \\
        -1
    \end{pmatrix*} = 0
    $$

    from which we obtain no further information beyond the fact that $x_1 = -x_2$ is a level curve of $f$. When 
    $x_1 > 0$, we have $\lambda_2 > 0$ so that the necessary (but not sufficient) conditions for a minimizer are 
    satisfied for $x_1 \in \mathopen( 0, \frac{1}{\sqrt{2}}\mathclose ]$. In particular, for the feasible direction 
    $p = (1, 1)^T$ and any $\epsilon > 0$ along $x_1 = -x_2$ (so that $f(x_1, x_2) = 0$), we have

    \begin{align*}
        f(x + \epsilon p) &= f(x_1 + \epsilon, x_2 + \epsilon) \\
                          &= x_1^2 + 2x_1 \epsilon + \epsilon^2 - (x_2^2 + 2x_2 \epsilon + \epsilon^2) \\
                          &= f(x_1, x_2) + 2 \epsilon (x_1 - x_2) \\
                          &= 2 \epsilon (x_1 - x_2).
    \end{align*}

    Hence $f(x + \epsilon p) > f(x)$ whenever $x_1 > x_2$ and $f(x + \epsilon p) < f(x)$ whenever $x_1 < x_2$. Since
    $x_1 \ge x_2$ along $x_1 = -x_2$ when $x_1 > 0$, we conclude that this segment of our active constraint is a
    local minimizer. A similar argument in the other direction shows that $x_1 = -x_2$ is a local maximizer for 
    $x_1 < 0$.

    At $x_1 = 0$, we simply have $x_1 = x_2 = \lambda_1 = \lambda_2 = f(x) = 0$. For feasible direction $p_1 = e_1$, we
    have


    \begin{align*}
        f(x + \epsilon p_1) &= f(x_1 + \epsilon, x_2) \\
                            &= x_1^2 + 2x_1 \epsilon + \epsilon^2 - x_2^2 \\
                            &= \epsilon^2 \\
    \end{align*}

    so that $f(x + \epsilon p_1) > f(x)$ for all $\epsilon > 0$. A similar argument for $p_2 = e_2$ shows that 
    $f(x + \epsilon p_2) = -\epsilon^2$ and hence $f(x + \epsilon p_2) < f(x)$ for all $\epsilon < 0$. The stationary 
    point $(0, 0)^T$ is therefore a saddle point. These results are confirmed graphically in Figure 
    \ref{fig:problem_4b}.

    \begin{figure}[h]
        \centering
        \includegraphics*[width=0.75\textwidth]{problem_4b.png}
        \caption{3d plot of $f(x)$ and constraints.}
        \label{fig:problem_4b}
    \end{figure}

    \pagebreak
    \paragraph{Case III:} $\lambda_2 = g_1 = 0.$ \ \\
    When $\lambda_2 = 0$, our first-order necessary condition becomes

    $$
    \nabla_x \mathcal{L} = \nabla f(x) - \lambda_1 \nabla g_1 = \begin{pmatrix*}[r]
        2 x_1 \\
       -2 x_2
    \end{pmatrix*} - \lambda_1 \begin{pmatrix*}[r]
        -2 \lambda_1 x_1 \\
        -2 \lambda_1 x_2
    \end{pmatrix*} = \begin{pmatrix*}
        0 \\
        0
    \end{pmatrix*}
    $$

    so that $2 x_1 (1 + \lambda_1) = 0$ and $2 x_2 (\lambda_1 - 1) = 0$. When $x_1 = 0$, our active constraint $g_1 = 0$
    requires that $x_2 = 1$ and we find $\lambda_1 = 1$. Similarly, when $x_2 = 0$, our constraints require $x_1 = 1$ and
    $\lambda_1 = -1$. For each stationary point, we have

    $$
    \nabla g \Big|_{(0, 1)} = \begin{pmatrix*}[r]
        0 & -2 \\
        1 &  1
    \end{pmatrix*} \quad \text{and} \quad \nabla g \Big|_{(1, 0)} = \begin{pmatrix*}[r]
       -2 &  0 \\
        1 &  1
    \end{pmatrix*}.
    $$
    
    Since both of these matrices are full rank, the reduced Hessian condition is trivially satisfied for minimizers and
    maximizers; for $x_* = (0, 1)$, we have $\lambda_1 > 0$ and hence $(0, 1)$ is a local minimizer with $f(x) = -1$. 
    Similarly, when $x_* = (1, 0)$, we have $\lambda_1 < 0$ and so $(1, 0)$ is a local maximizer where $f(x) = 1$.

    \paragraph{Case IV:} $\lambda_1 = \lambda_2 = 0.$ \ \\
    When all Lagrange multipliers are zero, our first-order necessary condition becomes

    $$
    \begin{pmatrix*}
        0 \\
        0
    \end{pmatrix*} = \nabla_x \mathcal{L} = \begin{pmatrix*}[r]
        2 x_1 \\
       -2 x_2
    \end{pmatrix*}
    $$

    from which we obtain $x_* = (0, 0)$ (and $f(x) = 0$) which addressed in \textbf{Case II}.

    \ \\\\
    Our optimal points therefore arise from \textbf{Case III}, whereby $(1, 0)$ maximizes the objective function with 

    $$
    f(1, 0) = 1^2 - 0 = 1
    $$

    and $(0, 1)$ minimizes the objective function with 
    
    $$
    f(0, 1) = 0 - 1^2 = -1.
    $$
    \ \\
\end{solution}