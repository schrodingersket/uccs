Consider an implicit (closed) curve in 2D

$$
f(x, y) = 0
$$

given by a particular parameterization $x = x(t)$, $y = y(t)$, where $t \in [0, L]$. For a choice of parameter values
$\{t_j\} = \{t_1, \dots, t_N\}$, consider the point cloud \linebreak
$\{ \bm{x}_j = \left( x_j, y_j \right) = \left( x_j(t), y_j(t) \right) \}$. Suppose we wish to reconstruct the implicit 
function from the given point cloud. Follow the following steps to interpolant the point cloud with a derivative-aware 
RBF:


\begin{enumerate}[(i)]
  \item Find an (outer) normal direction $\bm{n}_j$ to the curve at each point $\bm{x}_j$.
  \item Fix $\alpha > 0$ and compute the inner and outer points
        $$
          \bm{x}_j^- = \bm{x}_j - \alpha \bm{n}_j, \quad \bm{x}_j^+ = \bm{x}_j + \alpha \bm{n}_j.
        $$
  \item Interpolate the $3N$ data $\left( (\bm{x}_j^-, -\alpha), (\bm{x}_j, 0), (\bm{x}_j^+, \alpha) \right)$ using RBF interpolants, i.e., find a function $F: \mathbb{R}^2 \to \mathbb{R}$
        $$
          F(\bm{x}) = \sum\limits_{j=1}^{N} c_j \phi \left( \lVert \bm{x} - \bm{x}_j \rVert \right)
        $$

        such that $F(\bm{x}_j) = 0$, $F(\bm{x}_j^-) = -\alpha$, and $F(\bm{x}_j^+) = \alpha$.
  \item Restrict $F$ to the level set $F(x, y) = 0$ to obtain the RBF interpolation for the implicit curve $f(x, y) = 0$.
\end{enumerate}

\begin{solution}
  We consider the Fernandez-Guasti equation

  $$
  s^2 x^2 y^2 - \left( x^2 + y^2 \right) + 1 = 0,
  $$

  with squareness parameter $s$. When $s = 0$, the above equation represents a circle of radius 1; when $s = 1$, the 
  equation corresponds to a square of side length 2. When $0 < s < 1$, we obtain what is colloquially known as a 
  squircle. With an Elliptic Grid mapping, this becomes the system of parametric equations:\footnote{
    See \href{https://arxiv.org/vc/arxiv/papers/1604/1604.02174v1.pdf}{https://arxiv.org/vc/arxiv/papers/1604/1604.02174v1.pdf} 
    for details.
  }

  $$
  f(x, y) = \begin{cases}
    x(t) = \frac{1}{2} \left(2 + 2 \sqrt{2} \cos{t} + \cos{2t}\right)^{\frac{1}{2}} - \frac{1}{2} \left(2 - 2 \sqrt{2} \cos{t} + \cos{2t}\right)^{\frac{1}{2}}\\
    y(t) = \frac{1}{2} \left(2 + 2 \sqrt{2} \sin{t} - \cos{2t}\right)^{\frac{1}{2}} - \frac{1}{2} \left(2 - 2 \sqrt{2} \sin{t} - \cos{2t}\right)^{\frac{1}{2}}.
  \end{cases}
  $$

  \pagebreak
  We generate a point cloud with $s=0.95$ and random noise in the $x$ direction and compute normal directions $\bm{n}_j$ 
  at each point to obtain $\bm{x}^-_j$ and $\bm{x}^+_j$. We plot these points For $L = 2 \pi$ in Figure 
  \ref{fig:problem_1i}.

  \begin{figure}[h]
    \centering
    \includegraphics*[width=\textwidth]{problem_1i.png}
    \caption{Squircle point cloud}
    \label{fig:problem_1i}
  \end{figure}

  \pagebreak
  We utilize a multiquadric interpolant of the from

  $$
  \phi(r) = \sqrt{1 + (\epsilon r)^2}
  $$

  with shape parameter $\epsilon = 10$. In Figure \ref{fig:problem_1ii}, we project the point cloud onto the $xyz$-plane 
  and plot the RBF interpolant  of the now-3D point cloud. We see that the RBF interpolant correctly captures the squircular point cloud shape.

  \begin{figure}[h]
    \centering
    \includegraphics*[width=\textwidth]{problem_1ii.png}
    \caption{3D RBF interpolant}
    \label{fig:problem_1ii}
  \end{figure}

  \pagebreak
  Lastly, we project the RBF interpolant onto the $xy$-plane to find our interpolant for $f(x, y)$, in Figure 
  \ref{fig:problem_1iii}:

  \begin{figure}[h]
    \centering
    \includegraphics*[width=\textwidth]{problem_1iii.png}
    \caption{2D Squircular RBF Interpolant}
    \label{fig:problem_1iii}
  \end{figure}

  All calculations and plots are computed in \texttt{problem\_1.m}.
\end{solution}