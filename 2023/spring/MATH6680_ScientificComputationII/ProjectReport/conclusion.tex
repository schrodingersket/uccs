The results of Section \ref{sec:results} confirm that physics-informed neural networks can successfully be used to frame
the solution of non-trivial systems of partial differential equations as an optimization problem. Our PINN correctly 
inferred the shape and scale of an unknown bathymetry function for the inhomogeneous shallow-water equations. This 
result is, of course, only a small step in the general investigation of the efficacy of neural networks in solving 
inverse conservation problems. One area of further research is to explore the effect of various neural network 
architectures on the accuracy of inferred quantities. Other possibilities include the implementation of generalized 
conservation laws to recover results from simplified models (such as the SWE system examined here), extension to the 2D
case, modification of residual weights, and the application of PINNs to other conservation systems.

While traditional methods such as pseudospectral and finite 
difference solvers are typically faster (and more accurate) for well-defined problems, PINNs offer a flexible 
alternative for systems which may have missing or incomplete information but for which (possibly noisy) experimental 
data is available. As such, we believe that PINNs are particularly promising for modeling real physical systems for 
which the underlying physical mechanics are highly complex or not well understood. 

