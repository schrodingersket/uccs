In the Physics-Informed Neural Network approach, we build upon the methodology developed by Raissi et. al. 
\cite{raissi_physics-informed_2019} and approximate the solution to a system of partial differential equations with a
neural network. In particular, we seek neural network solution surrogates to (possibly nonlinear) equations of the form 

$$
u_t + \mathcal{N}[u; \lambda] = 0, \quad x \in \Omega, \quad t \in [0, T]
$$

where $\mathcal{N}$ is a parameterized spatial differentiation operator and $\Omega$ is a subset of $\mathbb{R}^n$. In 
the case of the shallow water equations, we have:

$$
\mathcal{N}[u; \lambda] = [F(u; \lambda)]_x - S(u; \lambda), \quad u = \begin{bmatrix}
    h \\
    v \\
    \alpha
\end{bmatrix}, \quad \text{and} \quad \lambda = C_f.
$$

We approximate the solution to the PDE by a neural network $U(x, t; \theta)$, where $\theta$ represents the parameters
of the neural network which is trained by minimizing the PDE residual. We define the residual 
$f \coloneqq u_t + \mathcal{N}[u]$ and, for given collocation points $\{x_i, t_i\}$, define the (mean-squared) loss 
function for the neural network in terms of $f$:

$$
MSE_f = \frac{1}{N_f} \sum_{i=1}^{N_f} |f(x_i, t_i)|^2
$$

Furthermore, when sampled data for the solution is available, we enforce consistency of the neural network with respect 
to measured data by introducing a second loss function for the sampled data:

$$
MSE_u = \frac{1}{N_u} \sum_{i=1}^{N_u} |U(x_i, t_i; \theta) - u(x_i, t_i)|^2
$$

where $u(x_i, t_i)$ represents a particular data sample and $U(x_i, t_i; \theta)$ the neural network prediction at that
sample point. We may therefore view the process of finding a neural network that satisfies the PDE as the unconstrained 
optimization problem:

\begin{mini*}
    {\theta}{MSE_f + MSE_u}{}{}
\end{mini*}

For this project, we generate an artificial data set for the $MSE_u$ term by first solving the PDE via pseudospectral 
methods and sampling the numerical solution at various temporospatial points. We train the neural network using DeepXDE
\cite{lu_deepxde_2021}, a Python library based on Tensorflow and PyTorch for solving PDEs via PINNs. 