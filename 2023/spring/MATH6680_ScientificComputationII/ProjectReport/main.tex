\documentclass[letterpaper,11pt]{article}
\usepackage[utf8]{inputenc}
\usepackage[bottom]{footmisc}
\usepackage{graphicx}
\usepackage{amsmath}
\usepackage{amsthm}
\usepackage{amscd}
\usepackage{amssymb}
\usepackage{latexsym}
\usepackage{upref}
\usepackage[hidelinks]{hyperref}
\usepackage{cite}
\usepackage{subcaption}
\usepackage{xfrac}
\usepackage{bm}
\usepackage{mathtools}
\usepackage{ifthen}
\usepackage{optidef}

\setlength{\textwidth}{6.4in}
\setlength{\textheight}{9.5in}
\setlength{\topmargin}{-1in}
\addtolength{\headheight}{0.0675in}
\setlength{\oddsidemargin}{0.2in}
\setlength{\evensidemargin}{0.2in}

\let\oldcite=\cite
\renewcommand\cite[1]{\ifthenelse{\equal{#1}{NEEDED}}{[citation~needed]}{\oldcite{#1}}}

\begin{document}

    \title{%
        1D Shallow-Water Wave Equation with Inferred Bathymetry\\
        \large A Physics-Informed Neural Network Approach
    }
    \author{%
        Nameika, Michael \\
        Thompson, Jonathan
    }
    \date{\today}
    \maketitle

    \begin{abstract}
        In this project, we examine a Physics-Informed Neural Network (PINN) approach to solving a 1D system of 
        shallow-water equations with variable bathymetry. We begin by stating the system of equations with variable
        bathymetry and from there derive the system of equations corresponding to a constant bathymetry with negligible
        frictional forces. From there, we solve the resulting system via pseudospectral methods for various boundary
        and initial conditions. We then use the results of this computation as sampled data with which to train a 
        Physics-Informed Neural Network (PINN); in this scheme, we treat the bathymetry as an unknown and allow the 
        neural network to infer the solution and bathymetry simultaneously. Lastly, we compare the results of the PINN
        approach to the results of the pseudospectral approach and discuss the advantages and disadvantages of each.
    \end{abstract}

    \section{Background}\label{sec:background}

    We first present the 1-D shallow-water equations for flood waves with variable bathymetry\cite{whitham1999_ch3}:

\begin{align*}
    h_t + (h v)_x &= 0 \\
    (h v)_t + \left[ hv^2 + \frac{1}{2} g h^2 \cos{(\alpha)} \right]_x &= g h \sin{(\alpha)} - C_f v^2
\end{align*}

where $v$ represents the average wave velocity, $h$ is the water depth, and $C_f$ is a friction
coefficient. Of these, $v$ and $h$ vary in time and space and while $C_f$ may, in general, vary with space we assume
it to be constant in this project. We expand the left-hand side of the second equation via the Chain Rule and 
substitute $h_t = -h_x v - h v_x$ from the first equation to find: 

\begin{align*}
    (h v)_t + \left[ hv^2 + \frac{1}{2} g h^2 \cos{(\alpha)} \right]_x 
        &= \left( h_t v + h v_t \right) + h_x v^2 + 2 h v v_x + \left[ \frac{1}{2} g h^2 \cos{(\alpha)} \right]_x \\
        &= -h_x v^2 - h v v_x + h v_t + h_x v^2 + 2 h v v_x + \left[ \frac{1}{2} g h^2 \cos{(\alpha)} \right]_x \\
        &=  h v v_x + h v_t + \left[ \frac{1}{2} g h^2 \cos{(\alpha)} \right]_x \\
        &=  h \left( v v_x + v_t + g h_x \cos{(\alpha)} + \frac{1}{2} g h [\cos{(\alpha)}]_x \right) \\
        &=  h \left( v v_x + v_t + g h_x \cos{(\alpha)} + g h [\cos{(\alpha)}]_x - \frac{1}{2} g h [\cos{(\alpha)}]_x \right) \\
        &=  h \left( v_t + \left[ \frac{v^2}{2} + g h \cos{(\alpha)} \right]_x - \frac{1}{2} g h [\cos{(\alpha)}]_x \right).
\end{align*}

\pagebreak
We may express our system in conservation form as

$$
    \textbf{u}_t + \left[ F(\textbf{u}) \right]_x = S(\textbf{u})
$$

where (after dividing through by $h$)

$$
\textbf{u} = \begin{pmatrix}
    h \\
    v
\end{pmatrix}, \quad F(\textbf{u}) = \begin{pmatrix}
    hv \\
    \frac{v^2}{2} + g h \cos{(\alpha)}
\end{pmatrix}, \quad S(\textbf{u}) = \begin{pmatrix}
    0 \\
    g \sin{(\alpha)} - C_f \frac{v^2}{h} + \frac{1}{2} g h [\cos{(\alpha)}]_x
\end{pmatrix}.
$$

We expand $[F(\textbf{u})]_x$ to find:

$$
[F(\textbf{u})]_x = \begin{pmatrix}
    h v_x + h_x v \\
    v v_x + g h_x \cos{(\alpha)} + g h [\cos{(\alpha)}]_x
\end{pmatrix} = \begin{pmatrix}
    v                & h \\
    g \cos{(\alpha)} & v
\end{pmatrix} \begin{pmatrix}
    h \\
    v
\end{pmatrix}_x + \begin{pmatrix}
    0 \\
    g h [\cos{(\alpha)}]_x
\end{pmatrix}
$$

and so our system becomes

$$
\textbf{u}_t + A(\textbf{u}) \textbf{u}_x = \tilde{S}(\textbf{u})
$$

where 

$$
A = \begin{pmatrix}
    v                & h \\
    g \cos{(\alpha)} & v
\end{pmatrix}, \quad \tilde{S} = \begin{pmatrix}
    0 \\
    g h \sin{(\alpha)} (1 - \alpha_x) - C_f \frac{v^2}{h}
\end{pmatrix}.
$$


    \section{Methodology}\label{sec:proposed-methodology}

    \subsection{Method of Characteristics}\label{subset:moc-methodology}

    We begin by solving the homogeneous system (i.e., we set $\tilde{S}(\textbf{u}) = 0$). The eigenvalues of $A$ are given by

\[\renewcommand\arraystretch{1.5}
\lambda = \begin{bmatrix}
    \lambda_1 \\
    \lambda_2
\end{bmatrix} = \begin{bmatrix}
    v + \sqrt{g h \cos{(\alpha)}} \\
    v - \sqrt{g h \cos{(\alpha)}}
\end{bmatrix}
\]

\noindent with corresponding right eigenvectors\footnote{
    We shall not use these immediately, but will need them later; it is convenient to introduce them alongside the 
    eigenvalues.
}

\[\renewcommand\arraystretch{1.5}
r_1 = \begin{bmatrix}
    \sqrt\frac{h}{g \cos{(\alpha)}} \\
    -1 \\
\end{bmatrix}, \quad r_2 = \begin{bmatrix}
    1 \\
    \sqrt\frac{h}{g \cos{(\alpha)}}
\end{bmatrix}.
\]

\noindent In the $x-t$ plane, our characteristic curves are defined by

\[\renewcommand\arraystretch{1.5}
\frac{d\textbf{x}}{dt} = \textbf{\lambda} (\textbf{u}(x(t), t)) = \begin{bmatrix}
    v + \sqrt{g h \cos{(\alpha)}} \\
    v - \sqrt{g h \cos{(\alpha)}}
\end{bmatrix}
\]

\noindent To determine our Riemann invariants, we seek out (currently unknown) quantitities $W_i$ which are conserved 
(i.e., constant) along some curve $\mathcal{C}$. In particular, we wish to find $\textbf{W}$ along our
characteristic curves, so that:

$$
\frac{d\textbf{W}}{dt} = 0
$$

\noindent along each curve defined in \textbf{x}. We examine the $i^{th}$ component $W_i$ of $\textbf{W}$ which we 
determine by setting the total derivative $\frac{d W_i}{dt} = 0$ along the $i^{th}$ characteristic curve (denoted by 
$x_i(t)$) corresponding to the $i^{th}$ element of \textbf{x}:

\begin{align*}
    0 &= \frac{d W_i}{dt} \\
      &= \frac{\partial W_i}{\partial t} + \frac{\partial W_i}{\partial x_i} \frac{dx_i}{dt} \\
      &= \frac{\partial W_i}{\partial h}\frac{\partial h}{\partial t}
       + \frac{\partial W_i}{\partial v}\frac{\partial v}{\partial t}
       + \left(\frac{\partial W_i}{\partial h}\frac{\partial h}{\partial x}
       + \frac{\partial W_i}{\partial v}\frac{\partial v}{\partial x} \right) \frac{dx_i}{dt} \\
      &= \nabla W_i \cdot \textbf{u}_t + \nabla W_i \cdot \textbf{u}_x \frac{dx_i}{dt}. \\
\end{align*}

\noindent We substitute $\frac{d x_i}{dt} = \lambda_i$ and $\textbf{u}_t = -A(\textbf{u}) \textbf{u}_x$ from the 
previous section to obtain:

$$
-\nabla W_i \left[ A(\textbf{u}) - I \lambda_i \right] \textbf{u}_x = 0.
$$

\noindent and so $-\nabla W_i (\textbf{u})$ must be a left eigenvector of $A(\textbf{u})$ corresponding to $\lambda_i$.
The vector $-\nabla W_i (\textbf{u})$ is therefore a right eigenvector of $A(\textbf{u})^T$, and so the matrix

$$
\begin{bmatrix}
    \vert                   & \cdots & \vert \\
    -\nabla W_1(\textbf{u}) & \cdots & -\nabla W_n(\textbf{u})  \\
    \vert                   & \cdots & \vert
\end{bmatrix} = -\textbf{J}(W)^T
$$

\noindent diagonalizes $A(\textbf{u})^T$, where $\textbf{J}(W)$ is the Jacobian of $W$. We define $L = -\textbf{J}(W)^T$
and write $A(\textbf{u})^T = L \Lambda L^{-1}$. Taking the transpose of both sides yields:

\begin{align*}
    A(\textbf{u}) &= \left( L \Lambda L^{-1} \right)^T   \\
                  &= \left(L^{-1}\right)^T \Lambda^T L^T \\
                  &= L^{-T} \Lambda L^T
\end{align*}

\noindent where $L^{-T}$ denotes the inverse transpose of $L$. We recall from the beginning of this section that the 
right eigenvectors of $A(\textbf{u})$ are given by:

$$
r_1 = \begin{bmatrix}
    \sqrt\frac{h}{g \cos{(\alpha)}} \\
    -1 \\
\end{bmatrix}, \quad r_2 = \begin{bmatrix}
    \sqrt\frac{h}{g \cos{(\alpha)}} \\
    1
\end{bmatrix}.
$$

\noindent and hence the matrix 

$$
R = \begin{bmatrix}
    \vert & \cdots & \vert \\
    r_1   & \cdots & r_n   \\
    \vert & \cdots & \vert
\end{bmatrix}
$$

\ \\
\noindent diagonalizes $A(\textbf{u})$ so that $A(\textbf{u}) = R \Lambda R^{-1}$. Combining this with the above result,
we obtain $L^T = R^{-1}$ and so:

\[\renewcommand\arraystretch{2}
-\textbf{J}(W) = L^T = R^{-1} = \begin{bmatrix}
    \sqrt{\frac{h}{g \cos{(\alpha)}}} & 1 \\
    -1                                & \sqrt{\frac{h}{g \cos{(\alpha)}}}
\end{bmatrix}^{-1} = \begin{bmatrix*}[r]
    \sqrt{\frac{g \cos{(\alpha)}}{4h}} & -\frac{1}{2} \\
    \sqrt{\frac{g \cos{(\alpha)}}{4h}} &  \frac{1}{2} \\
\end{bmatrix*}
\]

\noindent which at last allows us to explicitly compute the Riemann invariants, which we use to compute upwinded 
numerical flux at cell boundaries in the Discontinuous Galerken method: \footnote{
    We note that multiplying eigenvectors by a scalar value along the way has the effect of scaling the corresponding 
    Riemman invariants. Our Riemann invariants may therefore be equivalently be expressed \linebreak
    $W = (v - 2 \sqrt{gh \cos{\alpha}}, v + 2 \sqrt{gh \cos{\alpha}})^T$.
}

\[\renewcommand\arraystretch{2}
W = \begin{bmatrix}
    \frac{v}{2} - \sqrt{g h \cos{(\alpha)}} \\ 
    \frac{v}{2} + \sqrt{g h \cos{(\alpha)}} 
\end{bmatrix}.
\]


    \subsection{Pseudospectral Methods}\label{subsec:pseudospectral-methodology}

    To simulate experimental observations and generate a reference solution, we adopt the approach used by Fabien \cite{fabien2014spectral} and utilize a Fourier collocation method along 
with fourth-order Runge-Kutta time stepping. Recall the definition of the Fourier Transform of a differentiable function 
$f$:

$$
\mathcal{F}(f(x)) \equiv \int_{-\infty}^{\infty}f(x)e^{-ikx}dx
$$

where $\hat{f}(k) \equiv \mathcal{F}(f(x))$ is a function in $k$. From this, we also have the inverse Fourier transform:

$$
f(x) = \frac{1}{2\pi}\int_{-\infty}^{\infty} \hat{f}(k)e^{ikx}dk
$$

\noindent Differentiating both sides of the above equation with respect to $k$ yields

$$
\frac{df}{dx} = ik\frac{1}{2\pi}\int_{-\infty}^{\infty}\hat{f}(x)e^{ikx}dx
$$

\noindent and so

$$
\hat{f'}(k) = ik\hat{f}(k).
$$

Taking the inverse Fourier transform of the above equation yields the general form for the derivative of a function with 
respect to its Fourier transform:

$$
f'(x) = \mathcal{F}^{-1}(ik\mathcal{F}(f(x))).
$$

\noindent Extending this to the $n^{th}$ derivative yields:

$$
f^{(n)}(x) = \mathcal{F}^{-1}((ik)^n\mathcal{F}(f(x))).
$$

We apply these results to the SWE system with periodic boundary conditions, which takes the following form:

\begin{align*}
    \begin{bmatrix}
        h\\
        uh
    \end{bmatrix}_t
    + 
    \begin{bmatrix}
        uh\\
        hu^2 + \frac{gh^2}{2}
    \end{bmatrix}_x
    =
    \begin{bmatrix}
        0\\
        -ghB_x
    \end{bmatrix}
\end{align*}    

To compute the spatial derivative, we differentiate in the Fourier domain and apply the inverse Fourier transform 
$\mathcal{F}^{-1}$ to the resulting expression. Upon doing so, our system becomes

\begin{align*}
    \begin{bmatrix}
        h\\
        uh
    \end{bmatrix}_t
    + 
    \mathcal{F}^{-1}\left(ik\mathcal{F}\left(\begin{bmatrix}
        uh\\
        hu^2 + \frac{gh^2}{2}
    \end{bmatrix} \right)\right)
    = 
    \begin{bmatrix}
        0\\
        -ghB_x
    \end{bmatrix}.
\end{align*}

We approximate the time derivative with a fourth-order Runge-Kutta time-stepping scheme. This system is expressed in 
terms of the time derivatives of $u$ and $uh$; we therefore rewrite the second equation in the system in 
terms of $u$ and $uh$ to find:

$$
\renewcommand*{\arraystretch}{1.5}
\begin{bmatrix}
    h\\
    uh
\end{bmatrix}_t = -\begin{bmatrix}
    \mathcal{F}^{-1}(ik\mathcal{F}(uh))\\
    ghB_x + \mathcal{F}^{-1}(ik\mathcal{F}((uh)^2/h + gh^2/2)) 
\end{bmatrix} = \begin{bmatrix}
    F_1(u,uh)\\
    F_2(u,uh)
\end{bmatrix}.
$$

\noindent Given $h$ at step $n$, our update step to compute $h_{n+1}$ is

\begin{align*}
    k_{1,h} &= dt\frac{L}{2\pi}F_1(h_n, uh_n)\\
    k_{2,h} &= dt\frac{L}{2\pi}F_1(h_n + k_{1,h}/2, uh_n)\\
    k_{3,h} &= dt\frac{L}{2\pi}F_1(h_n + k_{2,h}/2, uh_n)\\
    k_{4,h} &= dt\frac{L}{2\pi}F_1(h_n + k_{3,h}, uh_n)\\
    h_{n+1} &= h_n + \frac{1}{6}(k_{1,h} + 2k_{2,h} + 2k_{3,h} + k_{4,h})
\end{align*}

\noindent As for the quantity $uh$, our update $(uh)_{n+1}$ is given by

\begin{align*}
    k_{1,uh} &= dt\frac{L}{2\pi}F_2(h_{n+1}, uh_n)\\
    k_{2,uh} &= dt\frac{L}{2\pi}F_2(h_{n+1}, uh_n + k_{1,uh}/2)\\
    k_{3,uh} &= dt\frac{L}{2\pi}F_2(h_{n+1}, uh_n + k_{2,uh}/2)\\
    k_{4,uh} &= dt\frac{L}{2\pi}F_2(h_{n+1}, uh_n + k_{3,uh})\\
    uh_{n+1} &= \frac{1}{6}(k_{1,uh} + 2k_{2,uh} + 2k_{3,uh} + k_{4,uh})
\end{align*}

where $dt$ is the time step and $\frac{L}{2\pi}$ the domain scaling factor. To simulate our desired experimental 
measurement data, we solve the SWE system subject to various (sinusoidal) initial conditions and bathymetry functions. 
In the homogeneous scheme, we impose a bathymetry function that is identically zero; for the inhomogeneous system, we
impose a sinusoidal bathymetry. The exact functions used for these computations are described in Section 
\ref{sec:results} and results of these computations for the homogenous system are shown in Figures 
\ref{fig:homogeneous_pseudospectral_swe_height} and \ref{fig:homogeneous_pseudospectral_swe_velocity}; inhomogeneous 
results are shown in Figure \ref{fig:inhomogeneous_pseudospectral_swe_velocity}.

    \subsection{Physics-Informed Neural Networks}\label{subsec:pinn-methodology}

    In the Physics-Informed Neural Network approach, we build upon the methodology developed by Raissi et. al. 
\cite{raissi_physics-informed_2019} and approximate the solution to a system of partial differential equations with a
neural network. In particular, we seek neural network solution surrogates to (possibly nonlinear) equations of the form 

$$
u_t + \mathcal{N}[u; \lambda] = 0, \quad x \in \Omega, \quad t \in [0, T]
$$

where $\mathcal{N}$ is a parameterized spatial differentiation operator and $\Omega$ is a subset of $\mathbb{R}^n$. In 
the case of the shallow water equations, we have:

$$
\mathcal{N}[u; \lambda] = [F(u; \lambda)]_x - S(u; \lambda), \quad u = \begin{bmatrix}
    h \\
    v \\
    \alpha
\end{bmatrix}, \quad \text{and} \quad \lambda = C_f.
$$

We approximate the solution to the PDE by a neural network $U(x, t; \theta)$, where $\theta$ represents the parameters
of the neural network which is trained by minimizing the PDE residual. We define the residual 
$f \coloneqq u_t + \mathcal{N}[u]$ and, for given collocation points $\{x_i, t_i\}$, define the (mean-squared) loss 
function for the neural network in terms of $f$:

$$
MSE_f = \frac{1}{N_f} \sum_{i=1}^{N_f} |f(x_i, t_i)|^2
$$

Furthermore, when sampled data for the solution is available, we enforce consistency of the neural network with respect 
to measured data by introducing a second loss function for the sampled data:

$$
MSE_u = \frac{1}{N_u} \sum_{i=1}^{N_u} |U(x_i, t_i; \theta) - u(x_i, t_i)|^2
$$

where $u(x_i, t_i)$ represents a particular data sample and $U(x_i, t_i; \theta)$ the neural network prediction at that
sample point. We may therefore view the process of finding a neural network that satisfies the PDE as the unconstrained 
optimization problem:

\begin{mini*}
    {\theta}{MSE_f + MSE_u}{}{}
\end{mini*}

For this project, we generate an artificial data set for the $MSE_u$ term by first solving the PDE via pseudospectral 
methods and sampling the numerical solution at various temporospatial points. We train the neural network using DeepXDE
\cite{lu_deepxde_2021}, a Python library based on Tensorflow and PyTorch for solving PDEs via PINNs. 

    \section{Comparison}\label{sec:comparison}
    
    \section{Conclusions and Future Work}\label{sec:conclusion}

    The following is a list of possible future directions for this work:

    \begin{enumerate}
        \item Allow the bathymetry to vary in space and solve the resulting system via Discontinuous Galerkin methods.
        \item Introduce frictional forces into the system and solve the resulting system via pseudospectral methods. In
              this paradigm, the friction coefficient will be treated as a parameter to be learned by the neural 
              network.
        \item Extend this problem to the two-dimensional case.
    \end{enumerate}

    \pagebreak

    \bibliographystyle{plain}
    \bibliography{refs}

    \pagebreak
    \appendix
    \section{Gauss-Legendre-Lobatto Quadrature}\label{appendix:gll}

In this section, we formulate an efficient method for computing Gauss-Legendre-Lobatto (GLL) quadrature points and 
weights via the Newton-Raphson method. We begin by defining the Legendre polynomial $P_n(x)$ of degree $n$ in terms of
solutions to Legendre's differential \linebreak
equation:\cite{NEEDED}

$$
(1 - x^2) L_n''(x) - 2 x L_n'(x) = -n (n + 1) L_n(x).
$$

\ \\
\noindent Bonnet's recursion formula is given by:\cite{NEEDED}

$$
\frac{x^2 - 1}{n} L_n'(x) = x L_n(x) - L_{n - 1}(x).
$$

Gauss-Legendre-Lobatto quadrature points are given by finding the roots of\linebreak
$(1 - x^2) L_n'(x)$ and since this expression is a polynomial, Newton-Raphson iteration is well-suited for this 
computation.\cite{Quarteroni2014} We therefore utilize the following recurrence relation to compute these roots:

\begin{equation}\label{eq:newton_raphson_gll}
x_{k+1} = x_k - \left[ \frac{d}{dx} \left[ (1 - x^2) L_n'(x) \right]\bigg\vert_{x_k} \right]^{-1} (1 - x_k^2)L_n'(x_k)
\end{equation}

where the denominator of the Newton step becomes (upon substitution of Legendre's differential equation)

\begin{align*}
\frac{d}{dx} \left[ (1 - x^2) L_n'(x) \right] &= (1 - x^2) L_n''(x) - 2 x L_n'(x) \\
                                              &= -n(n + 1) L_n(x).
\end{align*}

and the numerator of the Newton step is determined by Bonnet's recursion formula:

$$
(1 - x^2) L_n'(x) = n L_{n - 1}(x) - x n L_n(x).
$$

We substitute these expressions into (\ref{eq:newton_raphson_gll}) to obtain our iterative method for constructing the
desired GLL quadrature:

\begin{align*}
x_{k+1} &= x_k - \frac{n L_{n - 1}(x_k) - x_k n L_n(x_k)}{-n(n + 1) L_n(x_k)} \\
        &= x_k + \frac{L_{n - 1}(x_k) - x_k L_n(x_k)}{(n + 1) L_n(x_k)}.
\end{align*}

\end{document}
