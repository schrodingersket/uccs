\documentclass[letterpaper,11pt]{article}
\usepackage[utf8]{inputenc}
\usepackage[bottom]{footmisc}
\usepackage{graphicx}
\usepackage{amsmath}
\usepackage{amsthm}
\usepackage{amscd}
\usepackage{amssymb}
\usepackage{latexsym}
\usepackage{upref}
\usepackage[hidelinks]{hyperref}
\usepackage{cite}
\usepackage{subcaption}

\setlength{\textwidth}{6.4in}
\setlength{\textheight}{9.5in}
\setlength{\topmargin}{-1in}
\addtolength{\headheight}{0.0675in}
\setlength{\oddsidemargin}{0.2in}
\setlength{\evensidemargin}{0.2in}

\begin{document}

    \title{%
        1D Shallow-Water Wave Equation with Variable Bathymetry\\
        \large A Discontinuous Galerkin Approach
    }
    \author{%
        Thompson, J.
    }
    \date{\today}
    \maketitle

    \begin{abstract}
        In this project, we examine a Discontinuous Galkerin (DG) approach to solving a 1D system of shallow-water
        equations with variable bathymetry. We begin by stating the system of equations and from there, we derive the 
        discrete system of equations and discuss the numerical methods used to solve the system. We then present the 
        results of our numerical experiments, which include a comparison of the numerical solutions for a variety of
        boundary conditions and bathymetry functions.
    \end{abstract}

    \section{Background}\label{sec:background}

    We first present the 1-D shallow-water equations for flood waves with variable bathymetry\cite{whitham1999_ch3}:

    \begin{align*}
        h_t + (h v)_x &= 0 \\
        (h v)_t + \left[ hv^2 + \frac{1}{2} g h^2 \cos{(\alpha)} \right]_x &= g h \sin{(\alpha)} - C_f v^2
    \end{align*}

    where $v$ represents the average wave velocity, $h$ is the water depth, and $C_f$ is a friction
    coefficient. Of these, $v$ and $h$ vary in time and space and while $C_f$ may, in general, vary with space we assume
    it to be constant in this project. We expand the left-hand side of the second equation via the Chain Rule and 
    substitute $h_t = -h_x v - h v_x$ from the first equation to find: 
    
    \begin{align*}
        (h v)_t + \left[ hv^2 + \frac{1}{2} g h^2 \cos{(\alpha)} \right]_x 
            &= \left( h_t v + h v_t \right) + h_x v^2 + 2 h v v_x + \left[ \frac{1}{2} g h^2 \cos{(\alpha)} \right]_x \\
            &= -h_x v^2 - h v v_x + h v_t + h_x v^2 + 2 h v v_x + \left[ \frac{1}{2} g h^2 \cos{(\alpha)} \right]_x \\
            &=  h v v_x + h v_t + \frac{h}{h} \left[ \frac{1}{2} g h^2 \cos{(\alpha)} \right]_x \\
            &=  \frac{1}{h} \left[ v v_x + v_t + \frac{1}{3} \left[g h^3 \cos{(\alpha)} \right]_x + \frac{1}{6} g h^3 [\cos{(\alpha)}]_x \right] \\
            &=  \frac{1}{h} \left[ v_t + \left[ \frac{v^2}{2} + \frac{1}{3} g h^3 \cos{(\alpha)} \right]_x + \frac{1}{6} g h^3 [\cos{(\alpha)}]_x \right].
    \end{align*}

    \pagebreak
    We may therefore express our system in conservation form

    $$
        \textbf{u}_t + \left[ F(\textbf{u}) \right]_x = S(\textbf{u})
    $$

    where

    $$
    \textbf{u} = \begin{pmatrix}
        h \\
        v
    \end{pmatrix}, \quad F(\textbf{u}) = \begin{pmatrix}
        hv \\
        \frac{v^2}{2} + \frac{1}{3} g h^3 \cos{(\alpha)}
    \end{pmatrix}, \quad S(\textbf{u}) = \begin{pmatrix}
        0 \\
        g h \sin{(\alpha)} - C_f \frac{v^2}{h} - \frac{1}{6} g h^3 \sin{(\alpha)} \alpha_x
    \end{pmatrix}
    $$

    \section{Methodology}\label{sec:proposed-methodology}
    \subsection{Method of Characteristics}\label{subset:moc-methodology}

    \subsection{Physics-Informed Neural Networks}\label{subsec:pinn-methodology}

    \section{Comparison}\label{sec:comparison}

    \section{Conclusions and Future Work}\label{sec:conclusion}

    \pagebreak

    \bibliographystyle{plain}
    \bibliography{refs}


\end{document}
