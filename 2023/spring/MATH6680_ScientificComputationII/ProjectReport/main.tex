\documentclass[letterpaper,11pt]{article}
\usepackage[utf8]{inputenc}
\usepackage[bottom]{footmisc}
\usepackage{graphicx}
\usepackage{amsmath}
\usepackage{amsthm}
\usepackage{amscd}
\usepackage{amssymb}
\usepackage{latexsym}
\usepackage{upref}
\usepackage[hidelinks]{hyperref}
\usepackage{cite}
\usepackage{subcaption}
\usepackage{xfrac}
\usepackage{bm}
\usepackage{mathtools}
\usepackage{ifthen}

\setlength{\textwidth}{6.4in}
\setlength{\textheight}{9.5in}
\setlength{\topmargin}{-1in}
\addtolength{\headheight}{0.0675in}
\setlength{\oddsidemargin}{0.2in}
\setlength{\evensidemargin}{0.2in}

\let\oldcite=\cite
\renewcommand\cite[1]{\ifthenelse{\equal{#1}{NEEDED}}{[citation~needed]}{\oldcite{#1}}}

\begin{document}

    \title{%
        1D Shallow-Water Wave Equation with Variable Bathymetry\\
        \large A Discontinuous Galerkin Approach
    }
    \author{%
        Thompson, J.
    }
    \date{\today}
    \maketitle

    \begin{abstract}
        In this project, we examine a Discontinuous Galkerin (DG) approach to solving a 1D system of shallow-water
        equations with variable bathymetry. We begin by stating the system of equations and from there, we derive the 
        discrete system of equations and discuss the numerical methods used to solve the system. We then present the 
        results of our numerical experiments, which include a comparison of the numerical solutions for a variety of
        boundary conditions and bathymetry functions.
    \end{abstract}

    \section{Background}\label{sec:background}

    We first present the 1-D shallow-water equations for flood waves with variable bathymetry\cite{whitham1999_ch3}:

\begin{align*}
    h_t + (h v)_x &= 0 \\
    (h v)_t + \left[ hv^2 + \frac{1}{2} g h^2 \cos{(\alpha)} \right]_x &= g h \sin{(\alpha)} - C_f v^2
\end{align*}

where $v$ represents the average wave velocity, $h$ is the water depth, and $C_f$ is a friction
coefficient. Of these, $v$ and $h$ vary in time and space and while $C_f$ may, in general, vary with space we assume
it to be constant in this project. We expand the left-hand side of the second equation via the Chain Rule and 
substitute $h_t = -h_x v - h v_x$ from the first equation to find: 

\begin{align*}
    (h v)_t + \left[ hv^2 + \frac{1}{2} g h^2 \cos{(\alpha)} \right]_x 
        &= \left( h_t v + h v_t \right) + h_x v^2 + 2 h v v_x + \left[ \frac{1}{2} g h^2 \cos{(\alpha)} \right]_x \\
        &= -h_x v^2 - h v v_x + h v_t + h_x v^2 + 2 h v v_x + \left[ \frac{1}{2} g h^2 \cos{(\alpha)} \right]_x \\
        &=  h v v_x + h v_t + \left[ \frac{1}{2} g h^2 \cos{(\alpha)} \right]_x \\
        &=  h \left( v v_x + v_t + g h_x \cos{(\alpha)} + \frac{1}{2} g h [\cos{(\alpha)}]_x \right) \\
        &=  h \left( v v_x + v_t + g h_x \cos{(\alpha)} + g h [\cos{(\alpha)}]_x - \frac{1}{2} g h [\cos{(\alpha)}]_x \right) \\
        &=  h \left( v_t + \left[ \frac{v^2}{2} + g h \cos{(\alpha)} \right]_x - \frac{1}{2} g h [\cos{(\alpha)}]_x \right).
\end{align*}

\pagebreak
We may express our system in conservation form as

$$
    \textbf{u}_t + \left[ F(\textbf{u}) \right]_x = S(\textbf{u})
$$

where (after dividing through by $h$)

$$
\textbf{u} = \begin{pmatrix}
    h \\
    v
\end{pmatrix}, \quad F(\textbf{u}) = \begin{pmatrix}
    hv \\
    \frac{v^2}{2} + g h \cos{(\alpha)}
\end{pmatrix}, \quad S(\textbf{u}) = \begin{pmatrix}
    0 \\
    g \sin{(\alpha)} - C_f \frac{v^2}{h} + \frac{1}{2} g h [\cos{(\alpha)}]_x
\end{pmatrix}.
$$

We expand $[F(\textbf{u})]_x$ to find:

$$
[F(\textbf{u})]_x = \begin{pmatrix}
    h v_x + h_x v \\
    v v_x + g h_x \cos{(\alpha)} + g h [\cos{(\alpha)}]_x
\end{pmatrix} = \begin{pmatrix}
    v                & h \\
    g \cos{(\alpha)} & v
\end{pmatrix} \begin{pmatrix}
    h \\
    v
\end{pmatrix}_x + \begin{pmatrix}
    0 \\
    g h [\cos{(\alpha)}]_x
\end{pmatrix}
$$

and so our system becomes

$$
\textbf{u}_t + A(\textbf{u}) \textbf{u}_x = \tilde{S}(\textbf{u})
$$

where 

$$
A = \begin{pmatrix}
    v                & h \\
    g \cos{(\alpha)} & v
\end{pmatrix}, \quad \tilde{S} = \begin{pmatrix}
    0 \\
    g h \sin{(\alpha)} (1 - \alpha_x) - C_f \frac{v^2}{h}
\end{pmatrix}.
$$


    \section{Methodology}\label{sec:proposed-methodology}

    \subsection{Method of Characteristics}\label{subset:moc-methodology}

    We begin by solving the homogeneous system (i.e., we set $\tilde{S}(\textbf{u}) = 0$). The eigenvalues of $A$ are given by

\[\renewcommand\arraystretch{1.5}
\lambda = \begin{bmatrix}
    \lambda_1 \\
    \lambda_2
\end{bmatrix} = \begin{bmatrix}
    v + \sqrt{g h \cos{(\alpha)}} \\
    v - \sqrt{g h \cos{(\alpha)}}
\end{bmatrix}
\]

\noindent with corresponding right eigenvectors\footnote{
    We shall not use these immediately, but will need them later; it is convenient to introduce them alongside the 
    eigenvalues.
}

\[\renewcommand\arraystretch{1.5}
r_1 = \begin{bmatrix}
    \sqrt\frac{h}{g \cos{(\alpha)}} \\
    -1 \\
\end{bmatrix}, \quad r_2 = \begin{bmatrix}
    1 \\
    \sqrt\frac{h}{g \cos{(\alpha)}}
\end{bmatrix}.
\]

\noindent In the $x-t$ plane, our characteristic curves are defined by

\[\renewcommand\arraystretch{1.5}
\frac{d\textbf{x}}{dt} = \textbf{\lambda} (\textbf{u}(x(t), t)) = \begin{bmatrix}
    v + \sqrt{g h \cos{(\alpha)}} \\
    v - \sqrt{g h \cos{(\alpha)}}
\end{bmatrix}
\]

\noindent To determine our Riemann invariants, we seek out (currently unknown) quantitities $W_i$ which are conserved 
(i.e., constant) along some curve $\mathcal{C}$. In particular, we wish to find $\textbf{W}$ along our
characteristic curves, so that:

$$
\frac{d\textbf{W}}{dt} = 0
$$

\noindent along each curve defined in \textbf{x}. We examine the $i^{th}$ component $W_i$ of $\textbf{W}$ which we 
determine by setting the total derivative $\frac{d W_i}{dt} = 0$ along the $i^{th}$ characteristic curve (denoted by 
$x_i(t)$) corresponding to the $i^{th}$ element of \textbf{x}:

\begin{align*}
    0 &= \frac{d W_i}{dt} \\
      &= \frac{\partial W_i}{\partial t} + \frac{\partial W_i}{\partial x_i} \frac{dx_i}{dt} \\
      &= \frac{\partial W_i}{\partial h}\frac{\partial h}{\partial t}
       + \frac{\partial W_i}{\partial v}\frac{\partial v}{\partial t}
       + \left(\frac{\partial W_i}{\partial h}\frac{\partial h}{\partial x}
       + \frac{\partial W_i}{\partial v}\frac{\partial v}{\partial x} \right) \frac{dx_i}{dt} \\
      &= \nabla W_i \cdot \textbf{u}_t + \nabla W_i \cdot \textbf{u}_x \frac{dx_i}{dt}. \\
\end{align*}

\noindent We substitute $\frac{d x_i}{dt} = \lambda_i$ and $\textbf{u}_t = -A(\textbf{u}) \textbf{u}_x$ from the 
previous section to obtain:

$$
-\nabla W_i \left[ A(\textbf{u}) - I \lambda_i \right] \textbf{u}_x = 0.
$$

\noindent and so $-\nabla W_i (\textbf{u})$ must be a left eigenvector of $A(\textbf{u})$ corresponding to $\lambda_i$.
The vector $-\nabla W_i (\textbf{u})$ is therefore a right eigenvector of $A(\textbf{u})^T$, and so the matrix

$$
\begin{bmatrix}
    \vert                   & \cdots & \vert \\
    -\nabla W_1(\textbf{u}) & \cdots & -\nabla W_n(\textbf{u})  \\
    \vert                   & \cdots & \vert
\end{bmatrix} = -\textbf{J}(W)^T
$$

\noindent diagonalizes $A(\textbf{u})^T$, where $\textbf{J}(W)$ is the Jacobian of $W$. We define $L = -\textbf{J}(W)^T$
and write $A(\textbf{u})^T = L \Lambda L^{-1}$. Taking the transpose of both sides yields:

\begin{align*}
    A(\textbf{u}) &= \left( L \Lambda L^{-1} \right)^T   \\
                  &= \left(L^{-1}\right)^T \Lambda^T L^T \\
                  &= L^{-T} \Lambda L^T
\end{align*}

\noindent where $L^{-T}$ denotes the inverse transpose of $L$. We recall from the beginning of this section that the 
right eigenvectors of $A(\textbf{u})$ are given by:

$$
r_1 = \begin{bmatrix}
    \sqrt\frac{h}{g \cos{(\alpha)}} \\
    -1 \\
\end{bmatrix}, \quad r_2 = \begin{bmatrix}
    \sqrt\frac{h}{g \cos{(\alpha)}} \\
    1
\end{bmatrix}.
$$

\noindent and hence the matrix 

$$
R = \begin{bmatrix}
    \vert & \cdots & \vert \\
    r_1   & \cdots & r_n   \\
    \vert & \cdots & \vert
\end{bmatrix}
$$

\ \\
\noindent diagonalizes $A(\textbf{u})$ so that $A(\textbf{u}) = R \Lambda R^{-1}$. Combining this with the above result,
we obtain $L^T = R^{-1}$ and so:

\[\renewcommand\arraystretch{2}
-\textbf{J}(W) = L^T = R^{-1} = \begin{bmatrix}
    \sqrt{\frac{h}{g \cos{(\alpha)}}} & 1 \\
    -1                                & \sqrt{\frac{h}{g \cos{(\alpha)}}}
\end{bmatrix}^{-1} = \begin{bmatrix*}[r]
    \sqrt{\frac{g \cos{(\alpha)}}{4h}} & -\frac{1}{2} \\
    \sqrt{\frac{g \cos{(\alpha)}}{4h}} &  \frac{1}{2} \\
\end{bmatrix*}
\]

\noindent which at last allows us to explicitly compute the Riemann invariants, which we use to compute upwinded 
numerical flux at cell boundaries in the Discontinuous Galerken method: \footnote{
    We note that multiplying eigenvectors by a scalar value along the way has the effect of scaling the corresponding 
    Riemman invariants. Our Riemann invariants may therefore be equivalently be expressed \linebreak
    $W = (v - 2 \sqrt{gh \cos{\alpha}}, v + 2 \sqrt{gh \cos{\alpha}})^T$.
}

\[\renewcommand\arraystretch{2}
W = \begin{bmatrix}
    \frac{v}{2} - \sqrt{g h \cos{(\alpha)}} \\ 
    \frac{v}{2} + \sqrt{g h \cos{(\alpha)}} 
\end{bmatrix}.
\]


    \subsection{Physics-Informed Neural Networks}\label{subsec:pinn-methodology}

    \section{Comparison}\label{sec:comparison}

    \section{Conclusions and Future Work}\label{sec:conclusion}

    \pagebreak

    \bibliographystyle{plain}
    \bibliography{refs}

    \pagebreak
    \appendix
    \section{Gauss-Legendre-Lobatto Quadrature}\label{appendix:gll}

In this section, we formulate an efficient method for computing Gauss-Legendre-Lobatto (GLL) quadrature points and 
weights via the Newton-Raphson method. We begin by defining the Legendre polynomial $P_n(x)$ of degree $n$ in terms of
solutions to Legendre's differential \linebreak
equation:\cite{NEEDED}

$$
(1 - x^2) L_n''(x) - 2 x L_n'(x) = -n (n + 1) L_n(x).
$$

\ \\
\noindent Bonnet's recursion formula is given by:\cite{NEEDED}

$$
\frac{x^2 - 1}{n} L_n'(x) = x L_n(x) - L_{n - 1}(x).
$$

Gauss-Legendre-Lobatto quadrature points are given by finding the roots of\linebreak
$(1 - x^2) L_n'(x)$ and since this expression is a polynomial, Newton-Raphson iteration is well-suited for this 
computation.\cite{Quarteroni2014} We therefore utilize the following recurrence relation to compute these roots:

\begin{equation}\label{eq:newton_raphson_gll}
x_{k+1} = x_k - \left[ \frac{d}{dx} \left[ (1 - x^2) L_n'(x) \right]\bigg\vert_{x_k} \right]^{-1} (1 - x_k^2)L_n'(x_k)
\end{equation}

where the denominator of the Newton step becomes (upon substitution of Legendre's differential equation)

\begin{align*}
\frac{d}{dx} \left[ (1 - x^2) L_n'(x) \right] &= (1 - x^2) L_n''(x) - 2 x L_n'(x) \\
                                              &= -n(n + 1) L_n(x).
\end{align*}

and the numerator of the Newton step is determined by Bonnet's recursion formula:

$$
(1 - x^2) L_n'(x) = n L_{n - 1}(x) - x n L_n(x).
$$

We substitute these expressions into (\ref{eq:newton_raphson_gll}) to obtain our iterative method for constructing the
desired GLL quadrature:

\begin{align*}
x_{k+1} &= x_k - \frac{n L_{n - 1}(x_k) - x_k n L_n(x_k)}{-n(n + 1) L_n(x_k)} \\
        &= x_k + \frac{L_{n - 1}(x_k) - x_k L_n(x_k)}{(n + 1) L_n(x_k)}.
\end{align*}

\end{document}
