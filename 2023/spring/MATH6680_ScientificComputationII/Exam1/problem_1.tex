Describe and implement a fourth-order Runge-Kutta and Fourier method for Burger's equation 

$$
u_t = \epsilon u_{xx} + u u_x
$$

with periodic boundary conditions on the interval $\Omega = [-\pi, \pi]$ from $T = [0, 1]$ and initial condition

$$
u(x, 0) = e^{-10 \sin^2{\left( \frac{x}{2} \right)}}
$$

with $\epsilon = 0.03$.

\begin{solution}
  Since we impose periodic boundary conditions on our spatial domain, we perform Fourier spectral differentiation in 
  space and integrate in time via a fourth-order Runge-Kutta method. We first rewrite our PDE as follows:

  $$
  u_t - \epsilon u_{xx} - \left(\frac{1}{2} u^2 \right)_x = 0
  $$

  which upon Fourier transformation becomes

  $$
  \hat{u}_t + \epsilon k^2 \hat{u} - \frac{ik}{2} \widehat{u^2} = 0.
  $$

  We define $\widehat{V} \coloneqq e^{\epsilon k^2 t} \hat{u}$ (so that $\widehat{u} = e^{-\epsilon k^2 t} \widehat{V}$ 
  and $\widehat{V}_t \coloneqq e^{\epsilon k^2 t} \hat{u}_t + \epsilon k^2 e^{\epsilon k^2 t} \hat{u}$) and multiply 
  the previous equation through by our integrating factor $e^{\epsilon k^2 t}$ to obtain

  \begin{align*}
    0 &= e^{\epsilon k^2 t} \hat{u}_t + \epsilon k^2 e^{\epsilon k^2 t} \hat{u} - \frac{ik}{2} e^{\epsilon k^2 t} \widehat{u^2} \\
      &= \widehat{V}_t - \epsilon k^2 e^{\epsilon k^2 t} \hat{u} + \epsilon k^2 e^{\epsilon k^2 t} \hat{u} - \frac{ik}{2} e^{\epsilon k^2 t} \widehat{u^2} \\
      &= \widehat{V}_t - \frac{ik}{2} e^{\epsilon k^2 t} \widehat{u^2}.
  \end{align*}

  Moreover, since $\hat{u} = e^{-\epsilon k^2 t} \widehat{V}$, we have 
  $u = \mathcal{F}^{-1}\left( e^{-\epsilon k^2 t} \widehat{V} \right)$ and hence

  $$
  \widehat{V}_t = \frac{ik}{2} e^{\epsilon k^2 t} \mathcal{F} \left[ \mathcal{F}^{-1}\left( e^{-\epsilon k^2 t} \widehat{V} \right)^2 \right].
  $$

  Lastly, at $t = 0$ we have $\widehat{V} = \hat{u}$ so that our initial condition in $\widehat{V}$ is simply

  $$
  \widehat{V}_0 = \mathcal{F}(u(x, 0)) = \mathcal{F}\left( e^{-10 \sin^2{\left( \frac{x}{2} \right)}} \right).
  $$

  We integrate this system in time via a fourth-order Runge-Kutta method in \texttt{problem\_1.m} and show the results
  in Figure(\ref{fig:problem_1}).\footnote{
    In \texttt{problem\_1.m}, this problem is also solved without introducing the integration factor which yields the
    same results but is much less fun.
  }

  \begin{figure}[h]
    \centering
    \includegraphics*[width=.9\textwidth]{problem_1.png}
    \caption{Burger's equation solution for $\epsilon = 0.03$}
    \label{fig:problem_1}
  \end{figure}
  \ \\
\end{solution}