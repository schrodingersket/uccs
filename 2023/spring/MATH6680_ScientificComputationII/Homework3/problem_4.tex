\textbf{Trefethen 13.4}

The time step in \texttt{Program 37} is specified by $\Delta t = \frac{5}{\left( N_x + N_y^2 \right)}$. Study this
discretization theoretically and, if you like, numerically, and decide: Is this the right choice? Can you derive a more
precise stability limit on $\Delta t$?

\begin{solution}
    We perturb $\Delta t$ by letting $\Delta \tilde{t} = \frac{5 + \epsilon}{\left( N_x + N_y^2 \right)}$ to 
    numerically determine a more precise stability limit. With $\epsilon = 10$, we find that that solution instability
    is introduced at $t = 2$. We also find that heavy instabilities are introduced by setting the exponent of $N_y$ to 1
    in the denominator. These results are plotted in Figure \ref{fig:problem_4}.

    \begin{figure}[h]
        \centering
        \includegraphics*[width=.95\textwidth]{problem_4.png}
        \caption{Stability of various time steps for the wave equation.}
        \label{fig:problem_4}
    \end{figure}

    From Chapter 10, we recall that the Rule of Thumb\footnote{
        That is, that the eigenvalues of the linearized spatial discretization operator scaled by $\Delta t$ must lie in
        the stability region of the time-discretization operator.
    } for stability with Chebyshev points is $\frac{8}{N^2}$; hence we
    expect that the stability limit is at least $\Delta t = \frac{8}{N_y^2}$. Similarly, our Rule of Thumb indicates 
    that appropriate time steps for Fourier points in the leap frog discretization are at least 
    $\Delta t = \frac{6}{N_x}$. Combining these, we expect that a more precise stability limit for this problem is 

    $$
        \Delta t \le \min{\left\{ \frac{6}{N_x}, \frac{8}{N_y^2} \right\}}
    $$

    and so we let 

    $$
        \Delta t < \frac{6}{N_x + N_y^2}.
    $$
\end{solution}