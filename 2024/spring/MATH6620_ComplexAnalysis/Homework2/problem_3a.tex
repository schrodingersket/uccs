Prove that if $f:\mathbb{D} \to \mathbb{D}$ is holomorphic, then

$$
\rho(f(z), f(w)) \le \rho(z, w) \quad \text{for all} \; z, w \in \mathbb{D}.
$$

Moreover, prove that if $f$ is an automorphism of $\mathbb{D}$, then 

$$
\rho(f(z), f(w)) = \rho(z, w) \quad \text{for all} \; z, w \in \mathbb{D}.
$$

\ \\
\textit{Hint: } Consider the automorphism $\psi_{\alpha}(z) = \frac{z - \alpha}{1 - \conj{\alpha}{z}}$ and apply the 
Schwarz lemma to $\psi_{f(w)} \circ f \circ \psi_{w}^{-1}$.

\begin{solution}
  Let $w \in \mathbb{D}$. We first observe that $\psi_{f(w)} \circ f \circ \psi_{w}^{-1}: \mathbb{D} \to \mathbb{D}$ is 
  a composition of holomorphic functions and is therefore itself holomorphic. Moreover, by definition of 
  $\psi_{\alpha}$, we note that

  $$
  \psi_{f(w)} \circ f \circ \psi_{w}^{-1} (0) = 0
  $$

  and hence we may apply the Schwarz Lemma to conclude that

  $$
  \left| \psi_{f(w)} \circ f \circ \psi_{w}^{-1} (z) \right| \le |z|
  $$

  for every $z$ in the unit disc. Furthermore, since $\psi_w$ is bijective, there exists some unique 
  $\tilde{z} \in \mathbb{D}$ such that $\psi_w(\tilde{z}) = z$ and, in particular, $\tilde{z} = \psi_w^{-1}(z)$. Hence 
  we have

  $$
  \left| \frac{f(\tilde{z}) - f(w)}{1 - \conj{f(w)}f(\tilde{z})} \right| = \left| \psi_{f(w)} \circ f (\tilde{z}) \right| 
                                                                         = \left| \psi_{f(w)} \circ f \circ \psi_w^{-1}(z) \right| 
                                                                         \le |z| 
                                                                         = |\psi_w (\tilde{z})| 
                                                                         = \left| \frac{\tilde{z} - w}{1 - \conj{w} \tilde{z}} \right|
  $$

  Since $\psi_w$ is an automorphism of the unit disc, the fact that this expression holds for every $z \in \mathbb{D}$ 
  implies that it must also hold for every $\tilde{z} \in \mathbb{D}$ which shows that 
  $\rho(f(z), f(w)) \le \rho(z, w)$. We now suppose that $f$ is an automorphism of $\mathbb{D}$ so that 
  $f^{-1}:\mathbb{D} \to \mathbb{D}$ exists and is holomorphic. We apply the same argument to $f^{-1}$ to obtain

  $$
  \rho(f^{-1}(z), f^{-1}(w)) \le \rho(z, w)
  $$

  for every $z, w \in \mathbb{D}$. Moreover, since $f$ is an automorphism of $\mathbb{D}$, there again exist some unique 
  $\tilde{z}, \tilde{w} \in \mathbb{D}$ in the unit disc such that $f(\tilde{z}) = z$ and $f(\tilde{w}) = w$. From this,
  we conclude that

  $$
  \rho(\tilde{z}, \tilde{w}) \le \rho(f(\tilde{z}), f(\tilde{w}))
  $$

  and since this holds for every $\tilde{z}, \tilde{w} \in \mathbb{D}$, we conclude that 
  $\rho(z, w) = \rho(f(z), f(w))$ as desired.
  \ \\
\end{solution}