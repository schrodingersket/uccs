Prove that if $f : \mathbb{D} \to \mathbb{D}$ is analytic and has two distinct fixed points, then $f$ is the identity,
(i.e., $f(z) = z$ for all $z \in \mathbb{D}$).

\begin{solution}
  Let $w_1$ and $w_2$ be the two distinct fixed points of $f$, and let $\psi_{w_1}(w)$ be the conformal mapping defined
  by

  $$
  \psi_{w_1}(w) = \frac{w_1 - w}{1 - \overline{w_1}w}.
  $$

  Since $\psi_{w_1}$ is conformal, there exists a unique $\alpha \in \mathbb{D}$ such that 
  $\psi_{w_1}^{-1}(\alpha) = w_2$ (and $\alpha \neq 0$, since $\psi_{w_1}^{-1}(0) = w_1 \neq w_2$). We ``conjugate''
  $\psi_{w_1}$ by $f$ to obtain a new function $h: \mathbb{D} \to \mathbb{D}$ defined by

  $$
  h(w) = \psi_{w_1} \circ f \circ \psi_{w_1}^{-1}(w)
  $$

  and note that $h$ is holomorphic (as the composition of holomorphic functions) with $h(0) = 0$. Moreover, we have

  $$
  h(\alpha) = \psi_{w_1} \circ f \circ \psi_{w_1}^{-1}(\alpha)
            = \psi_{w_1} \circ f(w_2) 
            = \psi_{w_1} (w_2)
            = \alpha
  $$

  and hence $\alpha \neq 0$ is a fixed point of $h$. By the Schwarz lemma, we conclude that $h$ must be a rotation;
  however, since $h(\alpha) = \alpha$, we conclude that $h$ rotates $\mathbb{D}$ by an integer multiple of $2 \pi$ and 
  is therefore the identity. From this, we conclude in turn that $h(w) = w$ for all $w \in \mathbb{D}$, so that 
  $\psi_{w_1} \circ f \circ \psi_{w_1}^{-1}(w) = w$ and hence

  $$
  f \circ \psi_{w_1}^{-1}(w) = \psi_{w_1}^{-1}(w).
  $$

  Since this must hold for every $w \in \mathbb{D}$ and $\psi_{w_1}^{-1}(w)$ is an automorphism of the unit disc, we 
  conclude that $f(z) = z$ for all $z \in \mathbb{D}$, as desired.
  \ \\
\end{solution}