\textbf{Stein, Shakarchi 8.1}

A holomorphic mapping $f: U \to V$ is a \textbf{local bijection} on $U$ if for every $z \in U$ there exists an open disc
$D \subset U$ centered at $z$ such that $f: D \to f(D)$ is a bijection.

Prove that a holomorphic map $f : U \to V$ is a local bijection if and only if $f'(z) \neq 0$ for all $z \in U$.

\textit{Hint:} Use Rouché's theorem as in the proof of Proposition 1.1.

\begin{solution}
  We first show the reverse direction. Suppose that $f'(z) \neq 0$ for all $z \in U$ and let $z_0 \in U$ be arbitrary. 
  By translation if necessary (since translation does not affect whether $f$ is bijective or holomorphic), we may assume 
  that $z_0 = 0$ and that $f(z_0) = 0$. Since $f'(z) \neq 0$ for all $z \in U$, we have

  $$
  f(z) = a_1 z + g(z)
  $$

  where $g(z)$ vanishes to order $z^2$. We consider the function $h(z) = \frac{f(z)}{a_1}$ so that

  $$
  h(z) = z + g(z)
  $$

  where we have absorbed the constant $a_1 \neq 0$ into the definition of $g(z)$. From here, we observe that since
  $g(z)$ vanishes to order $z^2$, we may choose $z$ sufficiently close to zero (say, $|z| < \delta$) so that 
  $|g(z)| < C |z|^2$ for all $z \in B_{\delta}(0)$ in the open ball of radius $\delta$ centered at zero. Let 
  $w \in B_{\delta}(0)$. Then

  $$
  h(z) - w  = z - w + g(z),
  $$

  and for sufficiently small $w$ (say, $|w| < \epsilon < \delta$), we have $|z - w| > C|z|^2 \ge |g(z)|$. We now apply 
  Rouché's theorem on $B_{\epsilon}$ to conclude that $z - w$ and $z - w + g(z) = h(z) - w$ have the same number of 
  zeros (that is, one) whenever $w \in B_{\epsilon}(0)$. Lastly, we note that the preimage 
  $X = h^{-1}\left(B_{\epsilon}(0) \right)$ is open since $h$ is continuous and hence we have shown that $h$ is 
  injective with respect to $X$. Since $h$ is always surjective with respect to its range, we conclude that $h$ defines 
  a bijection between the open sets $X$ and $B_{\epsilon}(0)$, so that $h$ (and therefore $f$) is a local bijection, as 
  desired.


  To show the forward direction, we suppose that $f$ is a local bijection and fix $\zeta \in U$. Then there exists an open disc $D \subset U$ 
  centered at $\zeta$ such that $f: D \to f(D)$ is a bijection. Since $f$ is bijective on $D$, it is not constant and 
  hence $f(D)$ is open by the open mapping theorem.\footnote{
    At this point, we could simply apply Proposition 1.1 on p. 206 from \textit{Stein, Shakarchi} to $D$, but it was far 
    more illuminating to try to work out the small details in that proof and apply them here.
  }
  For the sake of contradiction, we suppose that $f'(z_0) = 0$ for some $z_0 \in D$; by making appropriate translations
  if needed, we may assume that $z_0 = 0$ and $f(z_0) = 0$. Then since 
  $f(z) = f(z_0) + a_1(z - z_0) + a_2(z - z_0)^2 + \hdots$, we obtain
  
  $$
    f(z) = a_1 z + a_2 z^2 + \hdots
  $$

  Since $f'(z_0) = f'(0) = 0$, we conclude that $a_1 = 0$ and hence $f(z) = z^k[a_k + z h(z)]$ for all $z$ in $D$ near 
  $0$ for some $k \geq 2$ such that $a_k \neq 0$, and where $h(z)$ is a non-vanishing holomorphic function in $D$. 

  Furthermore, we note that for some yet-to-be-chosen $w$ near $0 = f(z_0)$, we have

  $$
  f(z) - w = \left[a_k z^k - w\right] + z \cdot z^k h(z)
  $$

  We now choose $w$ so that $a_k z^k - w$ has at least two zeros in $D$ near $z_0$. To do this, we note that
  these zeros occur when $z = \left(\frac{w}{a_k}\right)^{1/k}$, and so we choose $w$ sufficiently small so
  that the set $\left\{z \in \mathbb{C} : |z| < \left(\frac{w}{a_k}\right)^{1/k}\right\}$ is contained 
  in $D$. Finally, since $D$ is an open connected subset of $\mathbb{C}$, we note that $h(z)$ must be bounded on $D$ by 
  the maximum modulus principle and hence we may choose $z$ sufficiently close to $0$ so that 
  $\left|z \cdot z^k h(z)\right| < \left|a_k z^k\right|$. By making $w$ sufficiently small again
  (and taking the minimum between it and our earlier choice of $w$ if necessary), we may ensure that
 
  $$
    \left|z \cdot z^k h(z)\right| < \left|a_k z^k - w\right|
  $$

  for all $z$ near $0$. With these conditions satisfied, we apply Rouché's theorem to conclude that $f(z) - w$ has at 
  least two zeros in $D$ near $0$. If these zeros are all the same (say, some $k \in D$), then 
  $f'(k) = 2 (k - k) g'(k) + (k - k)^2 g(k) = 0$, which contradicts 
  the fact that $f'(z) \neq 0$ for all $z$ sufficiently close to $0$. We therefore conclude that at least two zeros 
  are distinct.  Since $f(z)$ is non-constant on $D$, the zeros of $f(z) - w$ are isolated, and hence we have two 
  distinct points $z_1, z_2 \in D$ such that $f(z_1) = f(z_2) = w$ which contradicts the injectivity of $f$. With the 
  desired contradiction in hand, we conclude that $f(z) \neq 0$ for all $z \in \mathbb{D}$. Lastly, since $\zeta$ 
  was chosen to be arbitrary, $f'(z) \neq 0$ for every point $z \in U$, as was to be shown.
  \ \\
\end{solution}
