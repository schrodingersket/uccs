Prove that for $\lambda \in \mathbb{R} \setminus \{1\}$, the function

$$
f(z) = \int_0^z \frac{d\zeta}{\sqrt{\zeta (\zeta - 1)(\zeta - \lambda)}}
$$

is a conformal map from the upper half-plane $\mathbb{H}$ to a rectangle such that one of the rectangle's vertices is 
the image of the point at infinity.

\begin{solution} 
  Let $A_1 = 0$, $A_2 = 1$, and $A_3 = \lambda$ and let $\beta_1 = \beta_2 = \beta_3 = \frac{1}{2}$. We begin by 
  observing that $\beta_j < 1$ and

  $$
  1 < \sum_{k=1}^{3} \beta_k = \frac{3}{2} < 2,
  $$

  so that by Proposition 4.1, the image of the real axis under $f$ is a polygon of four sides with vertices 
  $a_1, a_2, a_3$, and $a_{\infty}$ where $a_j$ is the image of $A_j$, $a_{\infty}$ is the image of the point at 
  infinity, and the angle at the vertex $a_{\infty}$ is given by 
  
  $$
    \left[1 - \left(2 - \sum\limits_{k = 1}^3 \beta_k \right) \right] = \frac{\pi}{2}.
  $$

  Furthermore, the interior angles at $a_j$ are given by $\pi (1 - \beta_j) = \frac{\pi}{2}$, so that the image of the 
  real line under $f$ is a rectangle with vertices $a_1, a_2, a_3$, and $a_{\infty}$. It remains to show that $f$ is a 
  conformal map from $\mathbb{H}$ to a rectangle. By Theorems 4.6 and 4.7, we know that there exist complex numbers 
  $C_1 \neq 0$ and $C_2$ such that the conformal map $F$ of $\mathbb{H}$ to some rectangle $R$ is given by

  $$
  F(z) = C_1 \int_0^z \frac{d\zeta}{\sqrt{\zeta(\zeta - 1)(\zeta - \lambda)}} + C_2.
  $$

  where one of the vertices of the rectangle is the image of the point at infinity under $F$. In particular, since 
  translations and dilations of the image $F(\mathbb{H})$ have no effect on the conformality of $F$ or its domain, we 
  conclude that the function $f(z) = \frac{1}{C_1} F(z) - C_2$ must be a conformal map from $\mathbb{H}$ to a rectangle
  $R' = \frac{1}{C_1} R - C_2$, as was to be shown.
  
  \ \\
\end{solution}