\textbf{LeVeque 1.2} \\\\
a)  Use the method of undetermined coefficients to set up the $5\times5$ Vandermonde system which would determine a 
  fourth-order accurate finite difference approximation to $u''(x)$ based on five equally spaced points:

  $$
    u''(x) = c_{-2}u(x-2h) + c_{-1}u(x-h) + c_0u(x) + c_1u(x+h) + c_2u(x+2h) + \mathcal{O}(h^4)
  $$


\begin{solution}\ \\\\
    We begin by finding the Taylor expansion about each of our points; because our approximation contains five points,
    we need only determine the first five terms:

    \begin{align*}
    u(\bar{x} - 2h) &= u(\bar{x}) + \frac{(-2h)}{1!}u^{(1)}(\bar{x}) 
                                 + \frac{(-2h)^2}{2!}u^{(2)}(\bar{x}) 
                                 + \frac{(-2h)^3}{3!}u^{(3)}(\bar{x})
                                 + \frac{(-2h)^4}{4!}u^{(4)}(\bar{x})
                                 + \mathcal{O}(h^5) \\
    u(\bar{x} - h) &= u(\bar{x}) + \frac{(-h)}{1!}u^{(1)}(\bar{x}) 
                                 + \frac{(-h)^2}{2!}u^{(2)}(\bar{x}) 
                                 + \frac{(-h)^3}{3!}u^{(3)}(\bar{x})
                                 + \frac{(-h)^4}{4!}u^{(4)}(\bar{x})
                                 + \mathcal{O}(h^5) \\
    u(\bar{x}) &= u(\bar{x}) \\
    u(\bar{x} + h) &= u(\bar{x}) + \frac{(h)}{1!}u^{(1)}(\bar{x}) 
                                 + \frac{(h)^2}{2!}u^{(2)}(\bar{x}) 
                                 + \frac{(h)^3}{3!}u^{(3)}(\bar{x})
                                 + \frac{(h)^4}{4!}u^{(4)}(\bar{x})
                                 + \mathcal{O}(h^5) \\
    u(\bar{x} + 2h) &= u(\bar{x}) + \frac{(2h)}{1!}u^{(1)}(\bar{x}) 
                                 + \frac{(2h)^2}{2!}u^{(2)}(\bar{x}) 
                                 + \frac{(2h)^3}{3!}u^{(3)}(\bar{x})
                                 + \frac{(2h)^4}{4!}u^{(4)}(\bar{x})
                                 + \mathcal{O}(h^5) \\
    \end{align*}

    We now collect coefficients; since we are interested in an an approximation for $u^{(2)}(x)$, each
    collection of coefficients sums to zero save for that of the second derivative:

    \begin{align*}
    \begin{cases}
        c_{-2} + c_{-1} + c_0 + c_1 + c_2 = 0 \\
        (-2h)c_{-2} + (-h)c_{-1} + (h)c_1 + (2h)c_2 = 0 \\
        (2h^2)c_{-2} + (\frac{1}{2}h^2)c_{-1} + (\frac{1}{2}h^2)c_1 + (2h^2)c_2 = 0 \\
        (-\frac{4}{3}h^3)c_{-2} + (-\frac{1}{6}h^3)c_{-1} + (\frac{1}{6}h^3)c_1 + (\frac{4}{3}h^3)c_2 = 0 \\
        (\frac{2}{3}h^4)c_{-2} + (\frac{1}{24}h^4)c_{-1} + (\frac{1}{24}h^4)c_1 + (\frac{2}{3}h^4)c_2 = 0 \\
    \end{cases}
    \end{align*}

    Lastly, we divide each equation by an appropriate power of $h$ and write the result as a linear system, as desired:

    $$
    \begin{pmatrix}
        1            &           1  & 1 &           1 & 1 \\
        -2           &           -1 & 0 &           1 & 2 \\
        2            & \frac{1}{2}  & 0 & \frac{1}{2} & 2 \\
        -\frac{4}{3} & -\frac{1}{6} & 0 & \frac{1}{6} & \frac{4}{3} \\
        \frac{2}{3}  &           -1 & 0 & 1           & \frac{2}{3} \\
    \end{pmatrix}
    \begin{pmatrix}
        c_{-2} \\ c_{-1} \\ c_0 \\ c_1 \\ c_2
    \end{pmatrix}
        =
    \begin{pmatrix}
        0 \\
        0 \\
        \frac{1}{h^2} \\ 
        0 \\
        0 \\
    \end{pmatrix}
    $$
\end{solution}


b)  Compute the coefficients using the MATLAB code \texttt{fdstencil.m} available from the website, and check that they
  satisfy the system determined in part (a).

\begin{solution}\ \\\\
\end{solution}

c)  Test this finite difference formula to approximate $u''(1)$ for $u(x) = \sin(2x)$ with values of $h$ from the array 
  \texttt{hvals = logspace(-1, -4, 13)}. Make a table of the error vs.\ $h$ for several values of $h$ and compare 
  against the predicted error from the leading term of the expression printed by \texttt{fdstencil}. You may want to 
  look at the m-file \texttt{chap1example1.m} for guidance on how to make such a table.

  Also produce a log-log plot of the absolute value of the error vs.~$h$.  

  You should observe the predicted accuracy for larger values of $h$. For smaller values, numerical cancellation in 
  computing the linear combination of $u$ values impacts the accuracy observed.
    
\begin{solution}\ \\\\
\end{solution}