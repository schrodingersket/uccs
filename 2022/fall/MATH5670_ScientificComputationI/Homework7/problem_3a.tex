Observe how this behaves with $m + 1 = 50, 100,$ and 200 grid points. Change the final time to 
\texttt{tfinal = 0.1} and use the m-files \texttt{error\_table.m} and \texttt{error\_loglog.m} to verify second order 
accuracy.

\begin{solution}\ \\\\
    We observe approximately second order accuracy with the Lax-Wendroff method. The output of \texttt{problem\_3a.m} is
    given below: \ \\\\

    \begin{figure}[h]
        \begin{verbatim}
                  h        error       ratio       observed order
               0.02000   1.66622e-01       NaN             NaN
               0.01000   6.27543e-02   2.65514         1.40879
               0.00500   1.70902e-02   3.67195         1.87654
             
             
            Least squares fit gives E(h) = 108.653 * h^1.64267
        \end{verbatim}
        \caption{Output of \texttt{problem\_3a.m}}
    \end{figure}

    Since the least squares fit is given by $E(h) = 108.653 h^{1.64267}$, we see this method is roughly second order 
    accurate as desired. The \texttt{loglog} and solution plots for this method are given below:

    \begin{figure}[h]
        \centering
        \begin{subfigure}{0.45\textwidth}
            \includegraphics*[width=\textwidth]{problem_3a_error.png}
            \caption{\texttt{loglog} plot for the Lax-Wendroff method}
        \end{subfigure}
        \hfill
        \begin{subfigure}{0.45\textwidth}
            \includegraphics*[width=\textwidth]{problem_3a.png}
            \caption{Lax-Wendroff method solution}
        \end{subfigure}
    \end{figure}
\end{solution}