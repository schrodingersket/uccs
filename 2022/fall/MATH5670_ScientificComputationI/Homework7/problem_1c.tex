Show that this method is in fact stable for this range of Courant numbers by doing von Neumann analysis.

\textbf{Hint:} Let $\gamma(\xi) = e^{i \xi h} g(\xi)$ and show that $\gamma$ satisfies a quadratic equation closely
related to the equation (10.34) that arises from a von Neumann analysis of the leapfrog method.

\begin{solution}\ \\\\
    We proceed with von Neumann analysis by first making the following substitutions into our skewed leapfrog scheme:

    \begin{align*}
        U_j^n &= g^n(\xi) e^{i \xi jh} \\
        \nu &= \frac{ak}{h}
    \end{align*}

    Our scheme becomes:

    $$
    g(\xi)^{n+1} e^{i \xi jh} = g(\xi)^{n-1} e^{i \xi (j-2) h} - (\nu - 1) \left(g(\xi)^{n} e^{i \xi jh} - g(\xi)^{n} e^{i \xi (j-2) h}\right)
    $$

    We divide by $g(\xi)^{n-1} e^{i \xi (j-2) h}$ to find:

    $$
    \left[g(\xi) e^{i \xi h} \right]^2 = 1 - (\nu - 1)\left(e^{i \xi h} - e^{-i \xi h} \right)\left[ g(\xi) e^{i \xi h} \right]
    $$

    Letting $\gamma(\xi) = e^{i \xi h}g(\xi)$ yields a quadratic equation for $\gamma(\xi)$:

    $$
    \gamma(\xi)^2 = 1 - 2i(\nu - 1)\sin{(\xi h)}\gamma(\xi)
    $$
   
    If we define $z = \nu - 1$, we recover equation (10.34):

    $$
    \gamma(\xi)^2 = 1 - 2iz\sin{(\xi h)}\gamma(\xi)
    $$

    which yields the stability limit $|z| < 1$ and hence this scheme is stable for $|\nu - 1| < 1$. This is precisely 
    the range of Courant numbers $\nu = \frac{ak}{h}$ which satisfy the CFL condition in (b).
    \ \\
\end{solution}