Heun's 3rd order method:
\begin{center}
\begin{tabular}{c|ccc}
0 \\
1/3 & 1/3 \\
2/3 &  0  & 2/3\\
\hline \\
    & 1/4 &   0 & 3/4
\end{tabular}
\end{center}

\begin{solution}\ \\\\
    \textbf{(5.35)}

    \begin{flalign*}
    \sum_{j=1}^3 b_{j} &= \frac{1}{4} + 0 + \frac{3}{4} = 1 &\\
    \texttt{i=1: } &\sum_{j=1}^3 a_{1j} = 0 = c_1 &\\
    \texttt{i=2: } &\sum_{j=1}^3 a_{2j} = \frac{1}{3} = c_2 &\\
    \texttt{i=3: } &\sum_{j=1}^3 a_{3j} = 0 + \frac{2}{3} = \frac{2}{3} = c_3 &\\
    \end{flalign*}

    \textbf{(5.38)}

    \begin{flalign*}
    \sum_{j=1}^3 b_{j}c_{j} &= \frac{1}{4} \cdot 0 + 0 \cdot \frac{1}{3} + \frac{3}{4} \cdot \frac{2}{3} 
                             = \frac{1}{2} &
    \end{flalign*}
   
    \textbf{(5.39)}
    
    \begin{flalign*}
    \sum_{j=1}^3 b_{j}c_{j}^2 &= \frac{1}{4} \cdot 0^2 + 0 \cdot \left(\frac{1}{3}\right)^2 + \frac{3}{4} \cdot \left(\frac{2}{3}\right)^2 
                               = \frac{3}{4} \cdot \frac{4}{9} 
                               = \frac{1}{3} &
    \end{flalign*}

    \begin{flalign*}
    \sum_{i=1}^3\sum_{j=1}^3 b_{i}a_{ij}c_{j} &= \frac{1}{4} (0)
                                               + 0 \cdot \frac{1}{3} \cdot 0
                                               + \frac{3}{4} \left(0 + \frac{2}{3} \cdot \frac{1}{3} \right) 
                                               + \frac{3}{4} \cdot \frac{2}{9} 
                                               = \frac{1}{6} &
    \end{flalign*}

    Hence conditions (5.35), (5.38), and (5.39) are satisfied.
    \pagebreak

    To show that the Taylor series expansion of $e^{k \lambda}$ is recovered to order three, we apply one step of this
    method to $u' = \lambda u$:

    $$
    $$
    \ \\
\end{solution}