\textbf{LeVeque 2.2} 

a) Determine the Green's functions for the two-point boundary value problem $u''(x) = f(x)$ on $0 < x < 1$ with a 
   Neumann boundary condition at $x = 0$ and a Dirichlet condition at $x = 1$, i.e., find the function $G(x, \bar{x})$
   which solves

    $$
        u''(x) = \delta(x - \bar{x}),\; u'(0) = 0,\; u(1) = 0
    $$

   and the functions $G_0(x)$ solving

    $$
        u''(x) = 0,\; u'(0) = 1,\; u(1) = 0
    $$

   and $G_1(x)$ solving

    $$
        u''(x) = 0,\; u'(0) = 0,\; u(1) = 1.
    $$

\begin{solution}\ \\\\
    We first solve the system 
    
    $$
        u''(x) = \delta(x - \bar{x}),\; u'(0) = 0,\; u(1) = 0.
    $$

    Since $u''(x) = \delta(x - \bar{x})$, we know that $u''(x) = 0$ everywhere except at $x = \bar{x}$. Hence our 
    solution takes the form:

    $$
    u(x) = \begin{cases}
        m_1 x + b_1, \; x \le \bar{x} \\
        m_2 x + b_2, \; x \ge \bar{x}
    \end{cases}.
    $$


    Our Neumann condition on the left (i.e., $0 = x \le \bar{x}$) yields:

    $$
    0 = u'(0) = m_1
    $$
    
    and so $m_1 = 0$. Our Dirichlet condition at $x = 1$ (i.e., $1 = x \ge \bar{x}$) yields:

    $$
    0 = u(1) = m_2 + b_2.
    $$

    and hence $b_2 = -m_2$. Our system thus far is given by:

    $$
    u(x) = \begin{cases}
        b_1, \; x \le \bar{x} \\
        m_2 x - m_2, \; x \ge \bar{x}
    \end{cases}.
    $$
   
    \pagebreak

    Since $u(x)$ must be continuous, we have that the limit of $u(x)$ from the left and right
    at $x = \bar{x}$ must be equal, and hence:

    $$
    b_1 = m_2 \bar{x} - m_2.
    $$

    We solve for our final remaining unknown $m_2$ by integrating $u''(x) = \delta(x - \bar{x})$ over 
    $x \in [\bar{x} - \epsilon, \bar{x} + \epsilon]$ for some $\epsilon > 0$:

    \begin{align*}
        1 &= \int\limits_{\bar{x} - \epsilon}^{\bar{x} + \epsilon}{\delta(x - \bar{x})\, dx} \\
          &= \int\limits_{\bar{x} - \epsilon}^{\bar{x} + \epsilon}{u''(x)\, dx} \\
          &= u'(\bar{x} + \epsilon) - u'(\bar{x} - \epsilon).
    \end{align*}

    Hence the change in slope at $x = \bar{x}$ must be equal to one, and so we have:

    \begin{align*}
        1 &= u'_+(\bar{x}) - u'_-(\bar{x}) \\
          &= m_2 - 0 \\
          &= m_2.
    \end{align*}

    Our Green's function for this boundary value problem is therefore given by:

    $$
    G(x; \bar{x}) = \begin{cases}
        \bar{x} - 1,\; 0 \le x \le \bar{x} \\
        x - 1,\; \bar{x} \le x \le 1 \\
    \end{cases}.
    $$

    To find $G_0(x)$ and $G_1(x)$, we integrate both sides of $u''(x) = 0$ with respect to $x$ to find that both
    functions are linear. We first calculate $G_0(x)$. Since $G_0'(0) = 1$ we see that the slope $m$ of $G_0$ is equal
    to one. Our second initial condition at $G_0(1) = 0$ shows:
    
    $$
    0 = u(1) = m(1) + b = 1 + b
    $$

    and hence $b = -1$. $G_0(x)$ is therefore given by:

    $$
    G_0(x) = x - 1.
    $$

    Similarly, since $G_1'(x) = 0$, we know $G_1(x)$ is a constant; by our second initial condition $G_1(1) = 1$, we 
    know that $G_1(x)$ must be identically one:

    $$
    G_1(x) = 1.
    $$
\end{solution}

\pagebreak
b) Using this as guidance, find the general formulas for the elements of the inverse of the matrix in equation (2.54).
   Write out the $5 \times 5$ matrices $A$ and $A^{-1}$ for the case $h = 0.25$.

\begin{solution}\ \\\\
    Let $B$ denote the $(m+2) \times (m+2)$ inverse of $A$ from (2.54). Elements in the interior columns 
    (i.e., $1 \le i,j \le m$) are given by:

    \begin{align*}
    B_{ij} = hG(x_i;\; x_j) = \begin{cases}
                                x_j - 1,\; 0 \le x_i \le x_j \\
                                x_i - 1,\; x_j \le x_i \le 1 \\
                            \end{cases}.
    \end{align*}

    Elements in the first column (i.e, $j = 0$) of $B$ are given by:

    $$
    B_{i0} = G_0(x_i) = x_i - 1.
    $$

    Similarly, the elements in the last column (i.e., $j = m + 1$) of $B$ are given by:

    $$
    B_{i,m+1} = G_1(x_i) = 1.
    $$

    We let $h = 0.25$ and $m = 3$. From the above, we may write our matrix $B = A^{-1}$ explicitly:


    \begingroup
    \renewcommand*{\arraystretch}{1.5}
    $$
        B =A^{-1} =
        \begin{pmatrix}
                      -1 & -\frac{3}{16} &  -\frac{1}{8} & -\frac{1}{16} & 1 \\
            -\frac{3}{4} & -\frac{3}{16} &  -\frac{1}{8} & -\frac{1}{16} & 1 \\
            -\frac{1}{2} &  -\frac{1}{8} &  -\frac{1}{8} & -\frac{1}{16} & 1 \\
            -\frac{1}{4} & -\frac{1}{16} & -\frac{1}{16} & -\frac{1}{16} & 1 \\
                       0 &             0 &             0 &             0 & 1 \\
        \end{pmatrix}.
    $$
    \endgroup

    From (2.54), $A$ is given by:\footnote{
        See \texttt{problem\_2.m} for calculation of $B$ and verification that it's inverse is $A$ in MATLAB.
    }

    \begingroup
    \renewcommand*{\arraystretch}{1.5}
    $$
        A =
        \begin{pmatrix}
            -4 &   4 &   0  &   0 &  0 \\
            16 & -32 &  16  &   0 &  0 \\
             0 &  16 & -32  &  16 &  0 \\
             0 &   0 &  16  & -32 & 16 \\
             0 &   0 &   0  &   0 &  1 \\
        \end{pmatrix}.
    $$
    \endgroup
\end{solution}