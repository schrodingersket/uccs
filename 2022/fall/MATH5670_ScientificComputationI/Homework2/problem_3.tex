\textbf{LeVeque 2.3}
Determine the null space of the matrix $A^T$ where $A$ is given in equation (2.58), and verify that the condition (2.62)
must hold for the linear system to have solutions.

\begin{solution}\ \\\\
    The matrix $A^T \in M_{n \times n}$ where $A$ is given in equation (2.58) is given by:
    \[
    A^T = \begin{pmatrix} 
        -h &       1 \\
         h &      -2 & 1 \\
           &       1 & -2 & 1 \\
           &         & \ddots & \ddots & \ddots \\
           &         &        & 1      & -2 & 1 \\
           &         &        &        & 1  & -2 & h \\
           &         &        &        &    & 1  & -h \\
    \end{pmatrix}
    \]

    We reduce this matrix to row echelon form by adding (in order) the first row to the second, the second to the third,
    etc. The result of this process is the following matrix:
    
    \[
    A^T \rightarrow \begin{pmatrix} 
        -h &       1 \\
           &      -1 & 1 \\
           &         & -1     & 1 \\
           &         &        & \ddots & \ddots \\
           &         &        &        & -1 & 1 \\
           &         &        &        &    & -1 & h \\
           &         &        &        &    &    & 0 \\
    \end{pmatrix}
    \]

    The null space of this matrix (and hence $A^T$) is therefore the set of all vectors $\textbf{x} \in \mathbb{R}^n$ 
    which satisfy:

    \[
    \begin{pmatrix} 
        -h &       1 \\
           &      -1 & 1 \\
           &         & -1     & 1 \\
           &         &        & \ddots & \ddots \\
           &         &        &        & -1 & 1 \\
           &         &        &        &    & -1 & h \\
           &         &        &        &    &    & 0 \\
    \end{pmatrix} \textbf{x} = \textbf{0}.
    \]

    \pagebreak
    If we let $\textbf{x} = (x_1, x_2, ... x_n)^T$, then the above becomes the following system of $n-1$ equations and 
    $n$ unknowns:

    \begin{align*}
        \begin{cases}
                  -h x_1 + x_2 &= 0 \\
                    -x_2 + x_3 &= 0 \\
                             \vdots \\
            -x_{n-2} + x_{n-1} &= 0 \\
            -x_{n-1} + h x_{n} &= 0
        \end{cases}
    \end{align*}

    Hence $x_i = x_{i+1}$ for $2 \le i \le n-2$, $h x_1 = x_2$, and $h x_n = x_{n-1}$, and so (with $x_n$ as our free 
    variable), the solution to our system is given by:

    \[
    \begin{pmatrix} 
            x_1 \\
            x_2 \\
         \vdots \\
        x_{n-1} \\
            x_n \\
    \end{pmatrix} = 
    \begin{pmatrix} 
             1 \\
             h \\
        \vdots \\
             h \\
             1 \\
    \end{pmatrix} x_n
    \]

    The null space of $A^T$ is therefore the space spanned by $\beta = \{(1, h, ..., h, 1)^T\} \in \mathbb{R}^n$. To show that 
    condition (2.62) must hold for the linear system in (2.58) to have solutions, we need only show that $F$ is 
    orthogonal to the space spanned by $A^T$. Since $\beta$ forms a basis for the kernel of $A^T$, we need
    only show that $F$ is orthogonal to $\beta_1 = (1, h, ..., h, 1)^T$. Setting the dot product 
    $\langle \beta_1, F \rangle = 0$ yields:

    \[
    0 = \langle F, \beta_1 \rangle = \begin{pmatrix} 
             1 \\
             h \\
        \vdots \\
             h \\
             1 \\
    \end{pmatrix} \cdot
    \begin{pmatrix} 
             \sigma_0 + \frac{h}{2} f(x_0) \\
                                    f(x_1) \\
                                    \vdots \\
                                    f(x_m) \\
        -\sigma_1 + \frac{h}{2} f(x_{m+1}) \\
    \end{pmatrix} = \sigma_0 + \frac{h}{2} f(x_0) + \sum\limits_{k=1}^{m}{h f(x_k)} - \sigma_1 + \frac{h}{2} f(x_{m+1}).
    \]

    and hence

    $$
    \sigma_1 - \sigma_0 = \frac{h}{2} f(x_0) + \sum\limits_{k=1}^{m}{h f(x_k)} + \frac{h}{2} f(x_{m+1})
    $$

    as desired.
\end{solution}