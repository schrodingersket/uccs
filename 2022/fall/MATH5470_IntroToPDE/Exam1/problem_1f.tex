Suppose you have two bars and that the first one is much shorter than the second one. Describe how the behavior of the
two solutions differ. Why does this make sense?
\begin{solution}\ \\\\
    Given two bars where the first is much shorter than the second, the shorter bar will cool down much faster than the
    longer bar. The reasons for this are two-fold: the first is that the initial temperature contains an explicit 
    dependency on $x$ - thus, a shorter bar begins with a lower maximum temperature than the longer bar. As a result, 
    the longer bar cools from a higher initial temperature and as such we expect it to take longer than the short bar 
    to cool down to the steady state temperature. 

    The second reason for the accelerated cooling of the shorter bar has to do with the thermal energy contained in the
    bar; even if the initial temperature of the bars were identical, we'd expect that the longer bar cools more slowly
    due to the fact that it has more mass and therefore more thermal energy which must be lost to the environment 
    through the non-insulated end in order to reach a steady state.
    \ \\
\end{solution}