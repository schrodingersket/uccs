\documentclass[letterpaper,10pt]{article}
\usepackage[utf8]{inputenc}
\usepackage[bottom]{footmisc}
\usepackage{graphicx}
\usepackage{amsmath}
\usepackage{amsthm}
\usepackage{amscd}
\usepackage{amssymb}
\usepackage{latexsym}
\usepackage{upref}
\usepackage[hidelinks]{hyperref}
\usepackage{subcaption}

\setlength{\textwidth}{6.4in}
\setlength{\textheight}{9.5in}
\setlength{\topmargin}{-1in}
\addtolength{\headheight}{0.0675in}
\setlength{\oddsidemargin}{0.2in}
\setlength{\evensidemargin}{0.2in}

\begin{document}

    \title{
        Methods of Applied Mathematics: A Course Reflection\\
    }
    \author{%
        Thompson, J.
    }
    \date{\today}
    \maketitle

    We began the course by deriving the 2D polar form of the Laplace operator, starting with the Cartesian form and 
    making extensive use of the chain rule. This gave us the following forms of the Laplace operator:

    $$
    \Delta u = u_{xx} + u_{yy} = u_{rr} + \frac{1}{r} u_{r} + \frac{1}{r^2} u_{\theta\theta}.
    $$

    \section{Heat Equation}\label{sec:heat-equation}
    We quickly moved into the physical realm by considering heat flow through a uniform cylindrical rod. By making use
    of conservation laws for energy along with Gauss' law (also known as the divergence theorem), we arrived at the 
    one-dimensional heat equation, which is a classical example of a parabolic partial differential equation (PDE):

    $$
    u_{t} = \frac{k}{c \rho} u_{xx}
    $$

    \noindent where $u$ describes the temperature distribution in time, $k$ is the thermal conductivity of the material,
    $c$ is the specific heat capacity, and $\rho$ the material's density. We assumed that $\frac{k}{c \rho}$ is 
    constant, and in doing so found that we could utilize separation of variables by assuming a product solution of the
    form $u(x, t) = \phi(x) G(t)$ to find explicit solutions to the equation. We first examined steady-state solutions
    by letting $u_t = 0$ and solving the resulting system; typically, we'd find that steady state solutions are simply 
    linear functions. After that, we solved this equation with Dirichlet and Neumann boundary conditions; in the former
    case, temperature is prescribed along subregions of the domain boundary. In the latter case, the spatial derivative
    (that is, the rate of temperature flow) is prescribed along subregions of the domain boundary. We found in this case
    that our solutions typically exhibited damped oscillatory behavior by decaying toward the steady state solution over 
    time. In particular, separation of variables resulted in spatial ODE (the eigenvalue problem) and a temporal ODE; 
    the temperal ODE yielded an exponentially decaying term, and solutions to the spatial ODE gave rise to the 
    sinusoidal behavior described. As a result, we found that we'd need to determine valid eigenvalues along the way 
    (those which wouldn't lead to trivial solutions) as well as how to compute solution coefficients.
    
    \section{Fourier Series}\label{sec:fourier series}
    Computation of these solution coefficients gave rise to discussion regarding Fourier Series 
    (and corresponding convergence statements), wherein we discovered that arbitrary piecewise continuous functions may 
    be represented as linear combinations of the Schauder basis $\{\sin{(kx)}, \cos{(kx)} \}_{k \in \mathbb{Z}}$. By
    considering solutions to the heat equation as Fourier Series, we found that we could compute Fourier coefficients
    for solutions for a given function $f(x)$ (typically our initial condition as we solved the spatial ODE) by taking 
    advantage of the orthogonality of sines and cosines to derive explicit coefficient formulas:

    \begin{align*}
        f(x) &\sim a_0 \sum\limits_{k=1}^{\infty}{a_k \cos{\lambda x} + b_k \sin{\lambda x}} \\
        a_0 &= \frac{1}{2L} \int_{-L}^{L}{f(x)\; dx} \\
        a_n &= \frac{1}{L} \int_{-L}^{L}{f(x) \sin{\lambda x}\; dx} \\
        b_n &= \frac{1}{L} \int_{-L}^{L}{f(x) \cos{\lambda x}\; dx}
    \end{align*}

    For a domain $x \in [0, L]$, we found that these eigenvalues are given by $\lambda_n = \frac{n \pi}{L}$.

    \section{Laplace Equation}\label{sec:laplace-equation}
    With Fourier Series in hand, we directed our attention toward the 2D heat equation:

    $$
    u_t = \frac{k}{c \rho} \Delta u
    $$

    By considering only steady-state solutions to this equation, we arrived at Laplace's equation, a standard elliptic
    PDE:

    $$
    \Delta u = 0
    $$

    We solved this over rectangular and circular (and therefore elliptic, by way of appropriate variable 
    transformations) domains via separation of variables. We found that, in general, solutions in one spatial variable
    were periodic and that solutions in the other spatial variable must be exponential which we further learned is 
    generally true of harmonic equations.

    \section{Wave Equation}\label{sec:wave-equation}
    We next gave the heat equation a cold shoulder by considering a vibrating string. By analyzing the string's 
    displacement $u$ in terms of forces applied to an infinitesimal bit of string, we arrived at the classical 1D wave 
    equation which is perhaps the most canonical example of a hyperbolic PDE:

    $$
    u_{tt} = c^2 u_{xx}
    $$

    \noindent where $c$ is the wave velocity and is given by $c^2 = \frac{T}{\rho}$, where $T$ is the tension force 
    exerted on the string and $\rho$ is the linear density of the string. Separation of variables yielded the following
    solution to the wave equation:

    $$
    u(x, t) = \sum\limits_{k=1}^{\infty}{\sin{\left(\frac{k \pi x}{L}\right)} \left[ \alpha_k \cos{\left( \frac{c k \pi t}{L}\right)} + \beta_k \sin{\left( \frac{c k \pi t}{L}\right)}\right]}
    $$

    \noindent where the $\alpha_k$ term encodes contribution from the initial string displacement $u(x, 0) = f(x)$ and 
    $\beta_k$ encodes the solution contribution from the initial string velocity $u_t(x, 0) = g(x)$. After this, we 
    learned D'Alembert's delightful alternative analytic formulation of this solution, which is given by:

    $$
    u(x, t) = \frac{1}{2} \left[f(x - ct) + f(x + ct) \right] + \frac{1}{2c} \int_{x - ct}^{x + ct}{g(s)\; ds}.
    $$

    \ \\
    ...which brings us up to speed to where we are today.
\end{document}
