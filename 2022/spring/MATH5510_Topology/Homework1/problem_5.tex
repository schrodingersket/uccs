(\#8 a, b in 2.2) Let $f:X\to Y$ and $g:Y\to Z$ be any functions.

a. Prove that if $f$ is one-to-one and $g$ is one-to-one, then $g\circ f:X\to Z$
 is one-to-one. Is the converse true?\ \\

\begin{proof}\ \\\\
    Let $x_1, x_2 \in X$, and suppose that $g \circ f(x_1) = g \circ f(x_2)$.
    By the definition of function composition, we have that
    $g\left(f(x_1)\right) = g\left(f(x_2)\right)$. Hence $f(x_1) = f(x_2)$
    by injectiveness of $g$, and hence $x_1 = x_2$ by injectiveness of $f$. 
    Hence $g \circ f(x_1) = g \circ f(x_2)$ implies $x_1 = x_2$, and so
    $g\circ f:X\to Z$ is one-to-one.
    
    The converse, however, does not hold. To
    show this, it suffices to present a counterexample. Let
    $f:\mathbb{R}^2 \to \mathbb{R}^3$ and $g:\mathbb{R}^3 \to \mathbb{R}^2$ be
    the functions defined by $f(x, y) = (x, y, 0)$ and $g(x, y, z) = (x, y)$,
    respectively. Then $(g \circ f)(x, y) = (x, y)$ for all 
    $x,y \in \mathbb{R}$, and so $g \circ f = i_{\mathbb{R}^2}$. Hence
    $g \circ f$ is bijective (and therefore one-to-one). Observe, however, that
    $g(1, 2, 1) = g(1, 2, 2) = (1, 2)$, and hence $g$ is not one-to-one.
    \ \\ 
\end{proof}

\pagebreak
b. If $g$ is onto and $f$ is onto, then is $g\circ f$ always onto? Is the
   converse true?

\ \\
\emph{Claim.} If $g:Y \to Z$ is onto and $f:X \to Y$ is onto, then $g\circ f$ is
              also onto. The converse, however, is not true.

\begin{proof}\ \\\\
    Let $z$ be any element of $Z$. Since $g$ is onto, there exists at least one
    $y \in Y$ such that $g(y) = z$. Furthermore, since $f$ is onto and 
    $y \in Y$, there exists at least one $x \in X$ such that $f(x) = y$. Hence
    for any arbitary $z \in Z$, there exists an $x \in X$ such that
    $(g \circ f)(x) = z$, and hence $g \circ f$ is onto.

    To show that the converse does not hold, we present again the same
    counterexample from (a) and observe that because $g \circ f$ is bijective,
    $g \circ f$ must also be onto. Recall that $f(x, y) = (x, y, 0)$, and hence
    no $(x, y) \in \mathbb{R}^2$ can satisfy 
    $f(x, y) = (1, 1, 1) \in \mathbb{R}^3$. The function $f$ is therefore not
    onto, which proves our claim that the surjectiveness of $g \circ f$ does not
    imply that $f$ and $g$ are also themselves surjective.
    \ \\
\end{proof}