Prove that if $B\subseteq A$ and $B$ is infinite, then $A$ is infinite. 
Conclude that every subset of a finite set is finite.

\begin{proof}\ \\\\
    For the sake of contradiction, suppose that $A$ is finite. Observe that
    because $B$ is infinite, it is non-empty. Because $B$ is contained in $A$,
    we see that $A$ must therefore also be non-empty. Since $A$ is finite, there
    exists some $n \in \mathbb{N}$ such that $A \sim \{k\}_{k=1}^n$, and so
    there exists a bijection $\phi:A \to \{k\}_{k=1}^n$. We restrict $\phi$
    to $B$, and observe that because $\phi$ maps elements of $A$ to unique elements in 
    $\{k\}_{k=1}^n$ and because $\phi(B) \subseteq \phi(A)$, the image $\phi(B)$ is a set which
    also consists of distinct positive integers. We now construct a bijection $p:\phi(B) \to \{k\}_{k=1}^m$
     between these integers and a set $\{k\}_{k=1}^m, m \le n$.
    
    By the well-ordering principle, $\phi(B)$ contains a smallest element.
    We denote this element $b_1$, assign $p(b_1) = 1$, and observe that the set 
    $\phi(B) \backslash \{b_1\}$ is a proper subset of $\phi(B)$. Hence $\phi(B) \backslash \{b_1\}$
    is again a collection of distinct integers. A second invocation of the well-ordering
    principle on this set yields a new smallest element, which we denote $b_2$ and to which we assign 
    $p(b_2) = 2$.  Observe that because $B \subseteq A$ and Card$(A) = n$, proceeding in this 
    manner $n$ times must necessarily exhaust all elements in $\phi(B)$.  We therefore repeat this 
    process until all elements of $\phi(B)$ have been exhausted or until we have assigned $n$ such 
    elements to $p$. Lastly, we note that because $p$ assigns every element of $\phi(B)$ to a unique
    element $k \in \{k\}_{k=1}^m$ for some $m \le n$ where $m,n \in \mathbb{N}$, the function
    $p:\phi(B) \to \{k\}_{k=1}^m$ is a bijection from $\phi(B)$ to $\{k\}_{k=1}^m$. By (5), the
    composition $(p \circ \phi\vert_B):B \to \{k\}_{k=1}^m$ is a bijection from $B$ to a finite
    set of ordered natural numbers, which contradicts our assertion that $B$ is 
    infinite\footnotemark and $A$ therefore is not finite.

    To see that every subset of a finite set is finite, we simply consider the
    contrapositive of this proposition: If $B \subseteq A$ and $A$ is not
    infinite (i.e., it is finite), then $B$ is also not infinite (i.e., it is
    finite).
    
    \footnotetext{
        We apply the contrapositive of Theorem 2.5.2 from the text, i.e., if a 
        set $B$ is not empty and is not equivalent to $\{k\}_{k=1}^n$ for any
        $n \in \mathbb{N}$, then it is not finite.
    }
\end{proof}