Let $(X, d)$ be a metric space and let $Y$ be a subspace of $X$.

a.  Prove that $Z \subseteq Y$ is closed in $Y$ if and only if there exists a subset $A \subseteq X$ which is closed in
    $X$ such that $Z = A \cap Y$. 
\ \\\\
We begin by proving the following lemma. \ \\
\ \\
\emph{Claim.} If $D \subseteq E \subset F$, then $E \cap (F \setminus D) = E \setminus D$.
\begin{proof}\ \\\\
    If either side is empty, then $E$ and $D$ are disjoint and the result is immediate.
    Let $x \in E \cap (F \setminus D)$. Then $x$ is both in $E$ and not in $D$, and so $x \in E \setminus D$. Hence
    $E \cap (F \setminus D) \subseteq E \setminus D$.
    Conversely, if $x \in E \setminus D$, then $x$ is in $E$ and is not in $D$; since $E \subset F$, $x$ is also in $F$,
    and so $x \in E \cap (F \setminus D)$. Hence $E \setminus D \subseteq E \cap (F \setminus D)$, and so 
    $E \setminus D = E \cap (F \setminus D)$, as desired.
    \ \\
\end{proof}

\begin{proof}\ \\\\
    Suppose first that $A$ is closed in $X$ and that $Z = A \cap Y$. We first observe that the fact that $Z = A \cap Y$
    with our previous lemma yields the following:
    \begin{align*}
        Y \setminus Z &= Y \setminus (A \cap Y) \\
                      &= (Y \setminus A) \cup (Y \setminus Y) \\
                      &= Y \setminus A \\
                      &= Y \cap (X \setminus A).
    \end{align*}

    Then because $X \setminus A$ is open in $X$ (since $A$ is closed in $X$), we see that $Y \setminus Z$ may be 
    written as the intersection of $Y$ and a set which is open in $X$. Because $Y$ is a subspace of $Z$, we have that
    $Y \setminus Z$ must be open in $Y$, and so $Z$ is closed in $Y$, as desired.

    \ \\
    Suppose conversely that $Z$ is closed in $Y$. Then $Y \setminus Z$ is open in $Y$, and since $Y$ is a subspace of 
    $X$, there exists a set $B$ which is open in $X$ such that $Y \setminus Z = B \cap Y$. An application of our 
    previous lemma yields:
    \begin{align*}
        Z &= Y \setminus (Y \setminus Z) \\
          &= Y \setminus (B \cap Y) \\
          &= (Y \setminus B) \cup (Y \setminus Y) \\
          &= Y \setminus B \\
          &= Y \cap (X \setminus B).
    \end{align*}

    Lastly, because $X \setminus B$ is closed in $X$, we see that $Z$ may be written as the 
    intersection of the subspace $Y$ and a set which is closed in $X$, as desired.
    \ \\
\end{proof}

\pagebreak
b.  Show that every subset $Z \subseteq Y$ that is closed in $Y$ is closed in $X$ if and only if $Y$ is a closed subset
    of $X$.

\begin{proof}\ \\\\
    Suppose first that every subset which is closed in $Y$ is also closed in $X$. Since $Y$ itself is a subset of $Y$ 
    which is closed in $Y$, it is also closed in $X$.

    Now suppose that $Y$ is a closed subset of $X$, and let $Z \subseteq Y$ be an arbitrary subset of $Y$ which is 
    closed in $Y$. Then by part $(a)$, there exists a set $A \subseteq X$ which is closed in $X$ such that 
    $Z = A \cap Y$. Since the intersection of closed sets is itself closed, $A \cap Y$ is closed in $X$
    and hence $Z = A \cap Y$ is also closed in $X$. Because our choice of $Z$ was arbitrary, we see that every
    subset $Z \subseteq Y$ which is closed in $Y$ is also closed in $X$, as desired.
    \ \\
\end{proof}

\pagebreak
c.  Let $(X, d)$ be a metric space where $X = \mathbb{R}^2$ and $d$ is the Euclidean metric. Let 
    $Y = \{(x, 0) \mid x \in \mathbb{R} \} \subset X$ with the induced metric, and let 
    $Z = \{(x, 0) \mid 0 < x < 1 \} \subset Y$. Show that $Z$ is an open subset of $Y$ but that it is not an open subset
    of $X$.

\begin{proof}\ \\\\
    Let $B = \{(x, y) \mid [(x - \frac{1}{2})^2 + y^2 ]^{\frac{1}{2}} < \frac{1}{2} \}$. Then with respect to
    the Euclidean metric in $\mathbb{R}^2$, $B$ defines an open ball of radius $\frac{1}{2}$ centered about the point 
    $(\frac{1}{2}, 0)$, and is therefore open in $\mathbb{R}^2$. Then $B \cap Y = \{ (x, y) \mid 0 < x < 1 \} = Z$, and 
    hence $Z$ can be written as the intersection of $Y$ and a set $B$ which is open in $X$. Hence $Z$ is open in $Y$,
    as desired.

    We now show that $Z$ is not an open subset of $X$. Let $\epsilon > 0$ and $p = \left( \frac{1}{2}, 0 \right) \in Z$,
    and observe that any open ball $B_\epsilon(p)$ which is open in $\mathbb{R}^2$ also contains the point 
    $q = (\frac{1}{2}, \frac{\epsilon}{2})$. Since $\frac{\epsilon}{2} \neq 0$, $q$ is not in $Z$, and so no open ball
    centered about $p \in Z$ is entirely contained in $Z$. Hence $Z$ is not open.
    \ \\
\end{proof}

\pagebreak