a) Prove that every open connected subset of $\mathbb{R}^n$ is path connected.

\begin{proof}\ \\\\
    Let $A$ be an open connected subset of $\mathbb{R}^n$. If $A$ is empty then it is vacuously path connected, for any
    two points (of which there are none) may be connected by a path. We may therefore assume that $A$ is non-empty.

    Let $x$ be an arbitrary point in $A$, and let $G$ be the set of all points $p \in A$ for which there exists a path 
    between $p$ and $x$. Because each such point is in the open set $A$, there exists for each $p$ an open ball 
    $B_{r_p}(p) \subseteq A$ of radius $r_p > 0$ centered about $p$ which is entirely contained in $A$. Moreover, 
    because open balls are themselves path connected in $\mathbb{R}^n$ with the usual metric, each $B_{r_p}(p)$ is also
    a subset of $G$. Thus for each point in $p \in G$, there exists an open neighborhood about $p$ which is entirely 
    contained in $G$. Hence $G$ is open.

    Conversely, let $H$ be the set of all points $q \in A$ for which no path exists between $q$ and $x$, and observe
    that $H$ is the complement of $G$ with respect to $A$ and hence $G \cap H = \emptyset$. Because each $q \in H$ is 
    also in $A$, there exists for each $q$ an open ball $B_{r_q}(q) \subseteq A$ which is contained in $A$. Moreover,
    each open ball $B_{r_q}(q)$ must also be contained in $H$. To see this, suppose that there exists some point 
    $z \in B_{r_q}(q)$ such that a path exists which connects $z$ to $x$. Since open balls are path connected, a path 
    exists between $z$ and every point $q \in H$, and hence between $q$ and $x$. This contradicts the fact that points 
    in $H$ admit no path to $x$, and so no such $z$ exists. Each $B_{r_q}(q)$ is therefore contained in $H$ and 
    consequently $H$ is open.

    Lastly, recall that $G$ is non-empty by construction, and that $G$ and $H$ are open sets such that $G \cup H = A$.
    If $H$ is non-empty, then $G \cup H$ forms a disconnection of $A$, which contradicts the fact that $A$ is connected.
    $H$ is therefore empty, and so $G = A$. Moreover, because $G$ was defined to be the set of all points which admit a
    path to some $x \in A$, there exists a path between any two points in $G$ by way of $x$, and so $G$ (and therefore
    $A$) is path connected, as desired.
    \ \\
\end{proof}

\pagebreak


b) Prove that no continous real function on a closed interval $I \subset \mathbb{R}^2$ can be injective.

\begin{proof}\ \\\\
    Observe that any closed interval $I \subset \mathbb{R}^2$ is path connected, and let $f$ be a \linebreak
    continous real function. Because the continuous image of a connected set is connected, the image $f(I)$ is a 
    connected set in $\mathbb{R}$ and is therefore an interval.
    
    For the sake of contradiction, suppose that $f$ is injective, and let $x_1, x_2, x_3 \in I$ be three distinct points
    in $I$. Because $f$ is injective, $f(x_1)$, $f(x_2)$, and $f(x_3)$ represent three distinct points in the image 
    $f(I)$. Without loss of generality, suppose that \linebreak
    $f(x_1) < f(x_2) < f(x_3)$ and observe that because $f$ is injective, the set \linebreak
    $f(I) \setminus \{ f(x_1), f(x_2), f(x_3) \} \subset \mathbb{R}$ is not an interval and is therefore disconnected.
    
    To derive our desired contradcition, observe that the set 
    $I \setminus \{x_1, x_2, x_3\} \subset \mathbb{R}^2$ is path connected\footnote{
        This result follows from the fact that for any linear path between two points in $I$ which might intersect 
        $x_1$, $x_2$, or $x_3$, we may simply define a continuous piecewise linear path around these three missing 
        points as demonstrated in lecture.
    } and hence is also connected. Furthermore, because the restriction of a continuous function is itself continuous, 
    the restriction $\tilde{f} \coloneqq f\vert_{I \setminus \{x_1, x_2, x_3\}}$ \linebreak
    is continuous and so the continuous image
    $\tilde{f}(I \setminus \{x_1,x_2,x_3\}) = f(I) \setminus \{f(x_1),f(x_2),f(x_3)\}$
    is connected which contradicts our earlier assertion that $f(I) \setminus \{ f(x_1), f(x_2), f(x_3) \}$ is 
    disconnected. Hence $f$ cannot be injective, and since $f$ was chosen to be arbitrary, we see that no continuous 
    real function on a closed interval in $\mathbb{R}^2$ can injective.
    \ \\
\end{proof}
