Give an example of a pseudometric space.

    \ \\
    \emph{Claim.} The mapping
    $d: \mathbb{R} \times \mathbb{R} \rightarrow \mathbb{R}^+$ defined by 
    $d(x, y) \coloneqq \ln{\left(|x^2 - y^2| + 1\right)}, x,y \in \mathbb{R}$ 
    is a pseudometric and hence $(X, d)$ defines a pseudometric space for 
    $X = \mathbb{R}$.

    \begin{proof}\ \\\\
        Let $x, y, z \in \mathbb{R}$.
        
        Because $|x^2 - y^2| \ge 0$ and
        $\ln{x} \ge 0$ whenever $x \ge 1$, we have that
        $d(x, y) = \ln{\left(|x^2 - y^2| + 1\right)} \ge 0$ for all
        $x, y \in \mathbb{R}$. Moreover, when $x = y$:

        \begin{align*}
            d(x, y) &= \ln{\left(|x^2 - y^2| + 1\right)} \\
                    &= \ln{\left(|x^2 - x^2| + 1\right)} \\
                    &= \ln{1} \\
                    &= 0.
        \end{align*}

        Hence $d(x, y)$ satisfies the non-negative condition for a pseudometric. 
        Furthermore, for some $x \neq 0$ (and hence $x \neq -x$), observe that:

        \begin{align*}
            d(x, -x) &= \ln{\left(|x^2 - (-x)^2| + 1\right)} \\
                     &= \ln{\left(|x^2 - x^2| + 1\right)} \\
                     &= \ln{1} \\
                     &= 0.
        \end{align*}

        We therefore conclude that $d(x,y)$ is \emph{not} a metric.\footnote{
            Any metric is of course itself a pseudometric; the purpose of
            demonstrating that $d(x, y)$ is not a metric is purely pedagogical
            and is in no way required for $d(x, y)$ to satisfy the definition of
            a pseudometric.
        } $d(x, y)$ is, however, transitive:

        \begin{align*}
            d(x, y) &= \ln{\left(|x^2 - y^2| + 1\right)} \\
                    &= \ln{\left(|-(y^2 - x^2)| + 1\right)} \\
                    &= \ln{\left(|y^2 - x^2| + 1\right)} \\
                    &= d(y, x).
        \end{align*}

        \pagebreak
        Lastly, $d$ satisfies the Triangle Inequality:

        \begin{align*}
            d(x, z) + d(z, y) &= \ln{\left(|x^2 - z^2| + 1\right)}
                                 + \ln{\left(|z^2 - y^2| + 1\right)} \\
                              &= \ln{\left[
                                   \left(|x^2 - z^2| + 1\right)
                                   \cdot
                                   \left(|z^2 - y^2| + 1\right)
                                 \right]} \\
                              &= \ln{\left( 
                                   |x^2 - z^2| \cdot |z^2 - y^2|
                                 + |x^2 - z^2|
                                 + |z^2 - y^2|
                                 + 1
                                 \right)} \\
                              &= \ln{\left(
                                   |x^2 - z^2| \cdot |z^2 - y^2|
                                   + \Big| |x^2 - z^2| + |z^2 - y^2| \Big|
                                   + 1
                                 \right)} \\
                              &\ge \ln{\left(
                                   |x^2 - z^2| \cdot |z^2 - y^2|
                                   + |(x^2 - z^2) + (z^2 - y^2)|
                                   + 1
                                 \right)} \\
                              &= \ln{\left(
                                   |x^2 - z^2| \cdot |z^2 - y^2|
                                   + |x^2  - y^2|
                                   + 1
                                 \right)} \\
                              &\ge \ln{\left(
                                   |x^2  - y^2| + 1
                                 \right)} \\
                              &= d(x, y).
        \end{align*}

        Hence $d$ is pseudometric on $\mathbb{R}$ and $(X, d)$
        therefore forms a pseudometric space for $X = \mathbb{R}$.
        \\    

    \end{proof}