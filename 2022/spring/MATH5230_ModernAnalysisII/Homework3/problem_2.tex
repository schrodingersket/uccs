Let $(X, d)$ be a metric space and let $A \subset X$.

a.  Prove that $\bar{A}$ is the closure of $A$ if and only if $\bar{A}$ is the intersection of all closed subsets of $X$
    which contain $A$.

\begin{proof}\ \\\\
    Let $\bar{A}$ denote the closure of $A$, and let $U = \bigcap\limits_{i \in I}{U_i}$ denote the intersection of all 
    closed subsets of $X$ which contain $A$.\ \\

    Observe that $A \subseteq A \cup A' = \bar{A}$; $A$ is therefore contained in $\bar{A}$. Then since the closure of
    $A$ is a closed set, there exists some $k \in I$ for which $U_k = \bar{A}$, and so 
    $U = \bigcap\limits_{i \in I}{U_i} \subseteq U_k = \bar{A}$.

    To prove the reverse set inclusion, observe that because $U$ is the intersection of all closed sets which contain 
    $A$, $U$ itself must also contain $A$. Moreover, because the arbitrary intersection of closed sets is itself closed,
    $U$ is closed and therefore must also contain every limit point of $A$. Hence $\bar{A} = A \cup A' \subseteq U$, and
    so $\bar{A} = U$.
    \ \\
 \end{proof}

\pagebreak

b.  Show that $x \in \bar{A}$ if and only if $\inf\limits_{y \in A}{d(x, y)} = 0$.

\begin{proof}\ \\\\
    Suppose that $x \in \bar {A} = A \cup A'$. If $x \in A$, then $0 = d(x, x) \in \{ d(x, y) \mid y \in A\}$ and hence
    $\inf\limits_{y \in A}{d(x, y)} = 0$. If, on the other hand, $x \in A'$, then $x$ is a limit point of $A$ and every
    open ball $B_r(x)$ with radius $r > 0$ centered about $x$ contains a point in $A$. Hence for any fixed 
    $\epsilon > 0$, there exists a point $y \in A \cap B_{\epsilon}(x)$ and so for such a point, we have that 
    $d(x, y) < \epsilon$.  Since $\epsilon$ was chosen to be arbitrary, we see that 
    $\inf\limits_{y \in A}{d(x, y)} \le 0$; moreover, because $d(x, y) \ge 0$ for all $x, y \in X$, this inequality 
    becomes equality and thus $\inf\limits_{y \in A}{d(x, y)} = 0$. \ \\

    Conversely, suppose that $\inf\limits_{y \in A}{d(x, y)} = 0$. If $x$ is in $A$, then we have 
    $x \in A \subseteq A \cup A' = \bar{A}$; we may therefore assume that $x \in X \setminus A$. Because 
    $\inf\limits_{y \in A}{d(x, y)} = 0$, we have for any $\epsilon > 0$ that there must exist some point $y \in A$ such
    that $d(x, y) < \epsilon$. Hence $y$ is in the open ball $B_{\epsilon}(x)$ centered about $x$. Since this holds for 
    any $\epsilon > 0$, $x$ is a limit point of $A$, and hence $x \in A' \subseteq A \cup A' = \bar{A}$.
    \ \\
\end{proof}

\pagebreak

c.  The diameter $d(A)$ of a set $A$ in a metric space is defined to be $d(A) = \sup\limits_{x,y \in A}{d(x, y)}$. Prove
    that $d(A) = d(\bar{A})$. (Note that $d(A) < \infty$ if $A$ is bounded and $d(A) = \infty$ if $A$ is unbounded).

\begin{proof}\renewcommand{\qedsymbol}{}\ \\\\
    \begin{align*}
    \end{align*}
\end{proof}

\pagebreak