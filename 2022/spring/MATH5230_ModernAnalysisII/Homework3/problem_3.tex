Let $(X, d)$ be a metric space. Determine for each $A \subseteq X$ whether $A$ is open, closed, or neither in $X$.

a.  The set of integers $\mathbb{Z} \subset \mathbb{R}$ where $X = \mathbb{R}$ and $d$ is the usual metric.

\ \\
\emph{Claim.} $\mathbb{Z}$ is closed and not open.

\begin{proof}\renewcommand{\qedsymbol}{}\ \\\\
    Observe that $\mathbb{R} \setminus \mathbb{Z} = \bigcup\limits_{z \in \mathbb{Z}}{(z, z + 1)}$. Since each 
    $(z, z + 1)$ is open in $\mathbb{R}$ with respect to the usual metric, $\mathbb{R} \setminus \mathbb{Z}$ is the
    union of open sets and is therefore itself open. Since the complement of $\mathbb{Z}$ is open, $\mathbb{Z}$ itself
    is closed.

    Observe also that every open ball $B_\epsilon(1)$ centered about any $1 \in \mathbb{Z}$ contains a point in 
    $\mathbb{R} \setminus \mathbb{Q}$ by the density of the irrationals; hence there exists no open ball 
    $B_\epsilon(1) \subseteq \mathbb{Z}$, and so $\mathbb{Z}$ is not open.
    \ \\
\end{proof}

\pagebreak

b.  $A = \{(x, y) \in \mathbb{R}^2 \mid y=x^2, x \in \mathbb{Q} \}$ where $X = \mathbb{R}^2$ and $d$ is the Euclidean
     metric.

\ \\
\emph{Claim.} $A$ is neither closed nor open.

\begin{proof}\ \\\\
    Observe that the sequence of partial sums of the sequence $\{s_n\}$ defined by 
    $s_n = \sum\limits_{k = 0}^{n}{\frac{1}{k!}}$ is a sequence consisting of elements of $\mathbb{Q}$, but that
    $\{s_n\}$ converges to $e \in \mathbb{R} \setminus \mathbb{Q}$. Then the sequence $\{s_n, s_n^2\}$ is a sequence in
    of terms in $A$, and the element $(e, e^2)$ is a limit point\footnote{
        Since $f(x) = x^2$ is a continuous function in $(X, d)$, and $s_n$ converges to $e$, we have that 
        $f(s_n^2) = f(e^2)$.
    } of $A$ which is not in $A$. Hence $A$ is not closed.

    We first show that $A$ is not open, observe that the point $(0, 0)$ lies in $A$, and consider any open ball 
    $B_r((0, 0))$ with $r > 0$ centered about $(0, 0)$. Then the point $(\frac{r}{2}, \frac{r}{2})$ is in the open ball,
    but is not in $A$; hence there exists no open ball centered about $(0, 0)$ which is entirely contained in $A$, and 
    so $A$ is not open.
    \ \\
\end{proof}

\pagebreak

c.  $A = \{(x, y, z) \in \mathbb{R}^3 \mid x^2 + y^2 + z^2 + 2z = 0 \}$, where $X = \mathbb{R}^3$ and $d$ is the 
    Euclidean metric.

\ \\
\emph{Claim.} $A$ is closed and not open.

\begin{proof}\renewcommand{\qedsymbol}{}\ \\\\
    To show that $A$ is not open, consider the point $p = (0, 0, 0) \in A$, and observe that for any $\epsilon > 0$, the 
    open ball $B_\epsilon(p)$ centered about $p$ contains the point 
    $q = \left( \frac{\epsilon}{2}, \frac{\epsilon}{2}, \frac{\epsilon}{2} \right)$. Then because 
    $(\frac{\epsilon}{2})^2 + (\frac{\epsilon}{2})^2 + (\frac{\epsilon}{2})^2 + \epsilon > 0$, we have that $q$
    is not in $A$. Thus, there exists no open ball centered about $p \in A$ which is contained in $A$, and hence $A$ is
    not open.

    TODO: Show that $A$ is closed.

    \begin{align*}
    \end{align*}
\end{proof}

\pagebreak

d.  $A = \{ f \in X \mid 0 < f(x) < 1, x \in [a, b] \}$ where $d(f, g) = \sup\limits_{x \in [a, b]}{|f(x) - g(x)|}$ and 
    $X = C[a, b]$.

\ \\
\emph{Claim.} $A$ is open and not closed. 

\begin{proof}\renewcommand{\qedsymbol}{}\ \\\\
    We first show that $A$ is not closed. Observe that for every $n \in \mathbb{N}$, the function $f_n$ defined by 
    $f_n(x) = \frac{1}{n}$ defines a continuous function which takes values strictly between zero and one and so 
    $f_n \in A$ for all $n \in \mathbb{N}$. Moreover, for any $n \in \mathbb{N}$, we have the following:
    \begin{align*}
         d(f_n, 0) &= \sup\limits_{x \in [a, b]}{|f_n(x) - 0|} \\ 
                   &= \sup\limits_{x \in [a, b]}{|f_n(x)|} \\ 
                   &= \frac{1}{n}.
    \end{align*}

    Thus, for any fixed $\epsilon > 0$, we can find an $N \in \mathbb{N}$ such that $d(f_N, 0) < \epsilon$ and hence
    $f_N \in B_\epsilon(0)$. Since $f_N \neq 0$, we have that $g(x) = 0$ is a limit point of $A$ which is not in $A$, 
    and hence $A$ is not closed.

    TODO: Show that $f(x)$ is open.
\end{proof}

\pagebreak

e.  $A = \{ f \in X \mid \int_a^b{f(x) \,dx} = 0 \}$ where $d(f, g) = \sup\limits_{x \in [a, b]}{|f(x) - g(x)|}$ and 
    $X = C[a, b]$.

\ \\
\emph{Claim.} $A$ is closed and not open.

\begin{proof}\renewcommand{\qedsymbol}{}\ \\\\
    We first show that $A$ is not open. Observe that $X \setminus A = \{f \in X \mid \int_a^b{f(x) \,dx} \neq 0 \}$, and
    let $f_n$ be the constant function defined by $f_n(x) = \frac{1}{n(b - a)}$ for $n \in \mathbb{N}$. Then 
    $\int_a^b{f(x) \,dx} = \frac{1}{n} > 0$. As before, we have the following:


    \begin{align*}
         d(f_n, 0) &= \sup\limits_{x \in [a, b]}{|f_n(x) - 0|} \\ 
                   &= \sup\limits_{x \in [a, b]}{|f_n(x)|} \\ 
                   &= \frac{1}{n(b - a)}.
    \end{align*}

    Since $b - a > 0$ is constant, we may for any fixed $\epsilon > 0$ pick an $N \in \mathbb{N}$ to be large enough 
    (specifically, larger than $\frac{1}{\epsilon(b - a)}$) so that $d(f_N, 0) < \epsilon$. Hence $f_N$ is in
    the open ball $B_\epsilon(0)$ of radius $\epsilon$ centered about $g(x) = 0$. Since $\epsilon$ was chosen to be
    arbitrary, $g(x) = 0$ is a limit point of $X \setminus A$ which is not in $A$ (since $g(x) = 0$ integrates to zero 
    over any interval), and so $X \setminus A$ is not closed. Since $X \setminus A$ is not closed, $A$ is not open.

    TODO: Show that $A$ is not open.
\end{proof}

\pagebreak

f.  $A = \{ (x, y) \in X \mid y = f(x) \}$ where $X = \mathbb{R}^2$, $f:\mathbb{R} \to \mathbb{R}$ is continuous, and 
    $d$ is the Euclidean metric.

\ \\
\emph{Claim.} $A$ is closed and not open.

\begin{proof}\renewcommand{\qedsymbol}{}\ \\\\
    \begin{align*}
    \end{align*}
\end{proof}

\pagebreak