Let $M_n(\mathbb{R})$ denote the set of all real $n \times n$ matrices. For matrices
$A \coloneqq (a_{ij})$ and $B \coloneqq (b_{ij})$ in $M_n{(\mathbb{R})}$, define
the function $d(A, B) = \max\limits_{1 \le i,j \le n}{|a_{ij} - b_{ij}|}$.

a.  Show that $d(A, B)$ is a metric on $M_n(\mathbb{R})$. \ \\

    \begin{proof}\ \\\\
        Let $A, B, C \in M_n(\mathbb{R})$ and observe that because $d$ is the maximum of a set of absolute values in
        $\mathbb{R}$, $d$ is a non-negative real-valued function.
        
        We first show that $d$ is transitive:

        \begin{align*}
            d(A, B) &= \max\limits_{1 \le i,j \le n}{|a_{ij} - b_{ij}|} \\
                    &= \max\limits_{1 \le i,j \le n}{|-(b_{ij} - a_{ij})|} \\
                    &= \max\limits_{1 \le i,j \le n}{|b_{ij} - a_{ij}|} \\
                    &= d(B, A). 
        \end{align*}

        To show that $d(A, B) = 0$ if and only if $A = B$, we first suppose that $A = B$ and observe that
        $d(A, B) = \max\limits_{1 \le i,j \le n}{|a_{ij} - b_{ij}|} = \max\limits_{1 \le i,j \le n}{|a_{ij} - a_{ij}|} = 0$.
        
        Conversely, suppose that $d(A, B) = 0$. Then $d(A, B) = \max\limits_{1 \le i,j \le n}{|a_{ij} - b_{ij}|} = 0$
        and so by the definition of maximum, we have that $|a_{ij} - b_{ij}| \le 0$ for each $1 \le i, j \le n$. Since
        $|a_{ij} - b_{ij}| \ge 0$ trivially, we have that $|a_{ij} - b_{ij}| = 0$ and hence $a_{ij} = b_{ij}$ by a
        similar argument as from (1a). Since this is true for every $1 \le i, j \le n$, we have that $A = B$ as
        desired.
        
        Lastly, recall by the triangle inequality for the usual metric on $\mathbb{R}$ that 
        $|a - c| + |c - b| \ge |a - b|$ for any $a, b, c \in \mathbb{R}$. Hence for each $1 \le i, j \le n$, we have
        that $|a_{ij} - c_{ij}| + |c_{ij} - b_{ij}| \ge |a_{ij} - b_{ij}|$ and so the triangle inequality holds for $d$:

        \begin{align*}
            d(A, C) + d(C, B) &= \max\limits_{1 \le i,j \le n}{|a_{ij} - c_{ij}|} + \max\limits_{1 \le i,j \le n}{|c_{ij} - b_{ij}|} \\
                              &\ge \max\limits_{1 \le i,j \le n}{|a_{ij} - b_{ij}|} \\
                              &= d(A, B).
        \end{align*}
    \end{proof}

    \pagebreak

b.  Let $\{A^{(k)}\}_{k=1}^{\infty}$ be a sequence in $M_n(\mathbb{R})$. Prove that $\{A^{(k)}\}_{k=1}^{\infty}$ is a
    convergent sequence if and only if $\{a_{ij}^{(k)}\}_{k=1}^{\infty}$ is a convergent subsequence in $\mathbb{R}$.\ \\

    \begin{proof}\renewcommand{\qedsymbol}{}\ \\\\
        Suppose first that $\{A^{(k)}\}_{k=1}^{\infty}$ is a convergent sequence so that $\{A^{(k)}\}_{k=1}^{\infty}$ 
        converges to some $A \in M_n(\mathbb{R})$. Then for any fixed $\epsilon > 0$, there exists some 
        $N \in \mathbb{N}$ such that $\max\limits_{1 \le i,j \le n}{|a_{ij} - a^{(k)}_{ij}|} = d(A, A^{(k)}) < \epsilon$
        whenever $k > N$, where $a_{ij}^{(k)}$ represents the $i^{th}$ row and $j^{th}$ column of the $k^{th}$ element 
        of $\{A^{(k)}\}_{k=1}^{\infty}$. In particular, by the definition of the maximum element of a set, we have that 
        $|a_{ij} - a_{ij}^{(k)}| < \epsilon$ for any particular choice of $i$ and $j$. Since our choice of $\epsilon$ 
        was arbitrary, we have that $\lim\limits_{k \to \infty}{a_{ij}^{(k)} = a_{ij}}$ and hence 
        $\{a_{ij}^{(k)}\}_{k=1}^{\infty}$ is a convergent sequence in $\mathbb{R}$ with the usual metric.
        
        Conversely, suppose that each $\{a_{ij}^{(k)}\}_{k=1}^{\infty}$ represents a convergent sequence in $\mathbb{R}$
        with the usual metric so that each $\{a_ij^{(k)}\}_{k=1}^{\infty}$ converges to some $a_{ij}$. Then for any 
        fixed $\epsilon > 0$, there exists for each unique $i, j$ some $N_{ij} \in \mathbb{N}$ such that 
        $|a_{ij}^{(k)} - a_{ij}| < \epsilon$ whenever $k > N_{ij}$. Let $N = \max\limits_{1 \le i,j \le n}{\{N_{ij}\}}$.
        Then whenever $k > N$, we have that 
        $d(A, A^{(k)}) = \max\limits_{1 \le i,j \le n}{|a_{ij} - a^{(k)}_{ij}|} < \epsilon$. Lastly, we let 
        $A$ be the matrix defined by $A = \left(a_{ij}\right)$ and observe that $a_{ij} \in \mathbb{R}$ for all
        $1 \le i,j \le n$. Hence $A \in M_n(\mathbb{R})$, and so $\{A^{(k)}\}_{k=1}^{\infty}$ is a convergent sequence.
    \end{proof}

    \pagebreak

c.  Show that $\left(M_n(\mathbb{R}), d\right)$ is a complete metric space.

    \begin{proof}\ \\\\
        Let $\{A^{(k)}\}$ be a Cauchy sequence in $M_n(\mathbb{R})$. Then for any fixed $\epsilon > 0$, there exists
        some $N_0 \in \mathbb{N}$ such that $d(A^{(k)}, A^{(l)}) < \frac{\epsilon}{4}$ whenever $k,l > N_0$. In 
        particular, by the definition of the maximum element of a set, we see that 
        $|a_{ij}^{(k)} - a_{ij}^{(l)}| < \frac{\epsilon}{4}$ for any particular choice of $i$ and $j$ for all 
        $k,l > N_0$. Since our choice of $\epsilon$ was arbitrary, we have that ${\{a_{ij}^{(k)}\}}_{k=1}^{\infty}$ is a
        Cauchy sequence in $\mathbb{R}$ and hence converges to some $a_{ij} \in \mathbb{R}$ by the completeness of 
        $\mathbb{R}$ with respect to the usual metric. Moreover, because ${\{a_{ij}^{(k)}\}}_{k=1}^{\infty}$ converges 
        to $a_{ij}$, there exists some $N_{ij} \in \mathbb{N}$ such that $|a_{ij}^{(l)} - a_{ij}| < \frac{\epsilon}{4}$ 
        whenever $l > N_{ij}$. We therefore define $N = \max\limits_{1 \le i, j \le n}{\left\{N_0, N_{ij}\right\}}$ and
        observe that the following holds for all $1 \le i, j \le n$ whenever $k, l > N$:

        \begin{align*}
            |a_{ij}^{(k)} - a_{ij}| &\le |a_{ij}^{(k)} - a_{ij}^{(l)}| + |a_{ij}^{(l)} - a_{ij}| \\
                                    &\le \frac{\epsilon}{4} + \frac{\epsilon}{4} \\
                                    &= \frac{\epsilon}{2} \\
                                    &< \epsilon.
        \end{align*}

        Lastly, because the above $N$ holds for every $i, j$, we have that:
        
        \begin{align*}
            d(A^{(k)}, A) &= \max\limits_{1 \le i,j \le n}{|a_{ij}^{(k)} - a_{ij}|} \\
                          &\le \frac{\epsilon}{2} \\
                          &< \epsilon.
        \end{align*}

        The sequence ${\{A^{(k)}\}}_{k=1}^{\infty}$ therefore converges to $A$ with respect to $d$, and since $A$
        is comprised of entries in $\mathbb{R}$, $A$ is in $M_n(\mathbb{R})$. Since ${\{A^{(k)}\}}_{k=1}^{\infty}$ was 
        chosen to be an arbitrary Cauchy sequence in $M_n(\mathbb{R})$, we have that $(M_n(\mathbb{R}), d)$ is a 
        complete metric space.
        \ \\
    \end{proof}

    \pagebreak