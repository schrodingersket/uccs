Determine whether each of the following pairs $(X, d)$ define a metric space.

a.  $X = \mathbb{R}, d(x, y) = \sqrt{|x - y|}$ for $x, y \in X$. \ \\

    We first prove the following lemma:

    \emph{Lemma.} If $a, b$ are non-negative numbers in  $\mathbb{R}$, then
    $\sqrt{a} + \sqrt{b} \ge \sqrt{a + b}$.

    \begin{proof}\ \\\\
        Let $a,b$ be non-negative numbers in $\mathbb{R}$. Then the following inequality holds:

        \begin{align*}
            (\sqrt{a} + \sqrt{b})^2 &= (\sqrt{a})^2 + (\sqrt{b})^2 + \sqrt{ab} \\
                                    &= a + b + \sqrt{ab} \\
                                    &\ge a + b.
        \end{align*}

        Taking the square root of both sides\footnote{
            Because $f(x) = \sqrt{x}$ is a monotonically increasing function and because the quantities on both sides of the inequality are 
            positive, the inequality is preserved when taking the square root.
        } yields the desired inequality.
    \end{proof}

    We are now ready to prove the following claim:\ \\

    \emph{Claim.} $d(x, y) = \sqrt{|x - y|}$ defines a metric on $X$.

    \begin{proof}\ \\\\
        Let $x, y, z \in \mathbb{R}$.
       
        Observe that since $|x - y| =  |-(y - x)| = |y - x|$ for all $x, y \in \mathbb{R}$, we have that $d(x, y) = d(y, x)$.

        To show that $d(x, y) = 0$ if and only if $x = y$, we first let $x = y$ and observe that $d(x, y) = \sqrt{|x - y|} = \sqrt{|x - x|} = 0$.
        Now suppose $d(x, y) = 0$. Then $0 = \sqrt{|x - y|}$, and hence $|x - y| = 0$. We therefore have that $x - y = 0$ and $-(x - y) = 0$, and so
        $x = y$, as desired. 

        To prove that the triangle inequality holds for $d$, we observe by the above lemma that:
        
        \begin{align*}
            d(x, z) + d(z, y) &= \sqrt{|x - z|} + \sqrt{|z - y|} \\
                              &\ge \sqrt{|x - z| + |z - y|} \\
                              &\ge \sqrt{|x - z + z - y|} \\
                              &= \sqrt{|x - y|} \\
                              &= d(x, y).
        \end{align*}
    \end{proof}

    \pagebreak

b.  $X = \mathbb{R}, d(x, y) = |x| + |x - y| + |y|$ when $x \neq y$, and 
    $d(x, y) = 0$ when $x = y$ for $x, y \in X$. \ \\

    \emph{Claim.} $d(x, y) = |x| + |x - y| + |y|$ defines a metric on $X$.
    \ \\

    \begin{proof}\renewcommand{\qedsymbol}{}\ \\\\
    \end{proof}

    \pagebreak

c.  $X$ is the space of all Riemann integrable functions on $[a, b]$, and 
    $d(f, g)$ is the function defined by 
    $d(f, g) = \int_a^b{|f(x) - g(x)|dx}$ for $f, g \in X$. \ \\

    \emph{Claim.} 
    \ \\

    \begin{proof}\renewcommand{\qedsymbol}{}\ \\\\
    \end{proof}

    \pagebreak

d.  Let $(U, d_U)$ and $(V, d_V)$ be metric spaces, and define $X = U \times V$
    to be the set of ordered pairs $(u, v)$ with $u \in U$ and $v \in V$.
    Let $d\left((u_1, v_1), (u_2, v_2))\right)$ be the function defined by
    $d\left((u_1, v_1), (u_2, v_2)\right) 
     = \text{max}\left\{d_U(u_1, v_1), d_V(u_2, v_2)\right\}$. \ \\
    
     \emph{Claim.} 
    \ \\

    \begin{proof}\renewcommand{\qedsymbol}{}\ \\\\
    \end{proof}

    \pagebreak